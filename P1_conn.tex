\documentclass{amsart}%options: 11pt, A4paper, 

\usepackage{amssymb,amsfonts,amsmath,amsthm} 
\usepackage[left=1.5truein,right=1.5truein]{geometry}%options:
\usepackage{pstricks,xkeyval,pst-xkey,pst-3dplot,pstricks-add}
\usepackage{longtable}

\input{Commands}

\newcommand{\bispt}{\Spt_{S^1,\bbG_m}}
\newcommand{\symspt}{\Spt^{\Sigma}}
\newcommand{\Fpbar}{\overline{\bbF}_p}

\begin{document}

\section{Voevodsky's connectivity theorem for $\bbP^1$-spectra}

The following is from an email from Aravind to me concerning the proof
of the connectivity theorem for $\bbP^1$ spectra.
\begin{quotation}
  Most of the proof appears in Morel's Trieste notes (in the section
  on the homotopy t-structure for $P^1$-spectra).  First you prove a
  connectivity result for $P^1$-stable homotopy sheaves by using
  Morel's $S^1$-stable connectivity theorem and studying what happens
  under $G_m$ loops and $G_m$-suspension: suspension preserves
  connectivity, and Morel shows that taking $G_m$-loops has the effect
  of making a ``contraction''.  Then, you globalize this.
\end{quotation}

Our goal is to prove theorem 4.14 of \cite{Voev98}, which should be
restated in terms of $\bbP^1$-spectra. 

\begin{theorem}
  Let $(X,x)$ be a pointed smooth scheme over $\Spec(k)$ where $k$ is
  an infinite field. Let $(Y,y)$ be a pointed simplicial sheaf. Then
  for any $n< \dim(X)$ 
  \begin{equation*}
    \SH(k)(S^{n}\wedge \bbG_m^{m}\wedge \Sigma^{\infty}X, \Sigma^{\infty} Y) = 0.
  \end{equation*}

  In particular, if we take $X = S^0$, this theorem says 
  \begin{equation*}
    \SH(k)(S^n\wedge\bbG_m^m\wedge \unit, \Sigma^{\infty}Y) = \tilde{\pi}_n(Y)_{-m} = 0
  \end{equation*}
  whenever $n<0$. 
\end{theorem}

We can formulate the theorem by using homotopy sheaves. The theorem
says (in Morel's notation) that
$\tilde{\pi}_{n}(\Sigma^{\infty}Y)_m(X)$ vanishes whenever
$n < \dim(X)$.

The theorem is equivalent to the vanishing of the homotopy groups
$\pi_{n}(\Sigma^{\infty}Y)_m(X)$ when $n < \dim(X)$. (Is this right?)
So certainly by \cite[Example 5.2.2]{Mor03} we have
$\pi_n(\Sigma^{\infty}Y)_m = 0$ as a sheaf whenever $n < 0$. How to
show $\pi_n(\Sigma^{\infty}Y)_m(X) = 0$ when $0\leq n < \dim X$?


\section{Assumptions from previous lectures}

\subsection{Facts about Nisnevich topology}

\begin{proposition}\cite[2.4.1]{Mor04}
  Let $M$ be a sheaf of abelian groups on $\Sm/k$, and let
  $X\in \Sm/k$ with Krull dimension $d$. Then whenever $n > d$,
  $H^n_{Nis}(X;M) = 0$.
\end{proposition}

\begin{proposition}\cite[2.4.1]{Mor04}
  For any $X \in \Sm/k$, and for any $x \in X(k)$, there is an
  isomorphism of pointed sheaves of sets in the Nisnevich topology
  \begin{equation*}
    X/(X-\{x\}) \cong \bbA^n / (\bbA^n - \{0\}).
  \end{equation*}
\end{proposition}

\subsection{$S^1$-spectra}

\begin{definition}
  Let $\SH_s^{S^1}(k)$ denote the homotopy category associated to the
  projective model structure on Nisnevich sheaves of simplicial
  $S^1$-spectra on $\Sm/k$. (Is this the same as localizing the
  collection of $S^1$ spectra of spaces equipped with the proj. model
  str?)
\end{definition}

\begin{definition}
  Let $\SH^{S^1}(k)$ denote the localization of the model category
  associated to $\SH_s^{S^1}(k)$ at the collection of maps
  $E \wedge \bbA^1 \to E$.
\end{definition}

\begin{remark}
  Advantages to using alternate model category structures? E.g., use
  presheaves instead of sheaves? Use Hovey's method of stabilizing wrt
  Quillen functor $X \wedge -$? 
\end{remark}

\begin{definition}
  An $S^1$-spectrum $E$ is said to be $n$-connected if for any
  $m\leq n$, the homotopy sheaves $\pi_m(E)$ are trivial. 
\end{definition}

\begin{definition}
  Let $\frakC$ be a triangulated category. A $t$-structure on $\frakC$
  is a pair of full subcategories $(\frakC_{\geq 0}, \frakC_{\leq 0})$
  which satisfies 
  \begin{enumerate}
      \item
    $(\forall X \in \frakC_{\geq 0}) (\forall Y \in \frakC_{\leq 0})(
    \Hom_{\frakC}(X, Y[-1])=0$
      \item $\frakC_{\geq 0}[1] \subseteq \frakC_{\geq 0}$ and
    $\frakC_{\leq 0}[-1] \subseteq \frakC_{\leq 0}$ 

      \item for any $X \in \frakC$ there exists a distinguished triangle 
    \begin{equation*}
      Y \to X \to Z \to Y[1]
    \end{equation*}
    for which $Y \in \frakC_{\geq 0}$, $Z\in \frakC_{\leq 0}[-1]$..
  \end{enumerate}

  The heart of a $t$-structure is the full subcategory given by
  $\frakC_{\geq 0} \cap \frakC_{\leq 0}$. 
\end{definition}


\begin{definition}[$t$-structure on $\SH_s^{S^1}(k)$]
  Define $\SH_s^{S^1}(k)_{\geq 0}$ to be the full subcategory of
  $\SH_s^{S^1}(k)$ consisting of objects $E$ such that $\pi_n(E)=0$
  whenver $n<0$.
  
  Define $\SH_s^{S^1}(k)_{\leq 0}$ to be the full subcategory of
  $\SH_s^{S^1}(k)$ consisting of objects $E$ such that $\pi_n(E)=0$
  whenver $n>0$.
\end{definition}

\begin{theorem}
  The triple
  $(\SH_s^{S^1}(k), \SH_s^{S^1}(k)_{\geq 0}, \SH_s^{S^1}(k)_{\leq 0})$
  is a $t$-structure on $\SH_s^{S^1}(k)$. 
\end{theorem}

% \begin{definition}
%   An $S^1$-spectrum $F$ is $\bbA^1$
% \end{definition}

\begin{proposition}\cite[Lemma4.2.4]{Mor03}
  The functor
  $L^{\infty} : \Spt^{S^1}_s(k) \to \Spt^{S^1}_{s,\bbA^1}(k)$
  identifies the $\bbA^1$-localized $S^1$ stable homotopy category
  with the homotopy category of $\bbA^1$-local $S^1$ spectra.
\end{proposition}

\begin{theorem}[$S^1$ stable connectivity theorem]
  Let $E \in \SH_s^{S^1}(k)$, and suppose that whenever $n < 0$ the
  sheaf $\pi_n E = 0$. Then for all $n<0$, $\pi_n L_{\bbA^1}E = 0$.
\end{theorem}

\begin{theorem}
  The pair $(\SH_{\geq 0}^{S^1}(k), \SH_{\leq 0}^{S^1}(k))$ is a
  $t$-structure on the category $\SH^{S^1}(k)$. 
\end{theorem}

\begin{definition}
  Strictly $\bbA^1$ invariant sheaf of Abelian groups.

  If $M$ is strictly $\bbA^1$ invariant sheaf of groups, define the
  Eilenberg-MacLane spectrum $HM$ associated to it. 
\end{definition}

\begin{proposition}
  $HM$ is $\bbA^1$ local iff $M$ is strictly $\bbA^1$ invariant.
\end{proposition}

\begin{proposition}
  The heart of the homotopy $t$ structure is equivalent to the
  category of strictly $\bbA^1$ invariant sheaves. 
\end{proposition}



\section{Inverting $\bbG_m$; $\bbP^1$ spectra}

\subsection{$\bbG_m$ suspension and loops}

\subsection{Homotopy sheaves of $\underline{Hom}(\bbG_m, E)$}

\begin{definition}
  Contraction of a sheaf of pointed sets $G$ (or abelian groups $G$). $G_{-1}$
\end{definition}

\begin{theorem}\cite[Lemma 4.3.11]{Mor03}
  $\pi_n(\underline{Hom}(\bbG_m, E)) \to \pi_{n}(E)_{-1}$ is iso. 
\end{theorem}

\begin{lemma}
  If $M$ is a strictly $\bbA^1$ invariant sheaf of abelian groups, then
  \begin{equation*}
    \underline{Hom}(\bbG_m, HM) \cong H(M_{-1})
  \end{equation*}
\end{lemma}

\begin{lemma}
When $n\neq 0$,
  \begin{equation}
    [\Sigma^{\infty}\bbG_m, HM[n]]_s^{S^1} = 0.
  \end{equation}

The following map is an iso. 
  \begin{equation}
    [\Sigma^{\infty}\bbG_m, HM]_s^{S^1} \to [S^0, H(M_{-1})]_s^{S^1}
  \end{equation}
\end{lemma}

\begin{thebibliography}{OOOO}
    \bibitem[A-1974]{Adams} Adams, J.F., {\it Stable Homotopy and
    Generalized Homology.} Chicago Lectures in Mathematics, (1974).

    % \bibitem[A-1965]{Adams-Periodicity} Adams, J.F., {\it A
    % periodicity theorem in homological
    % algebra}. Proc. Camb. Phil. Soc. (1966).

    \bibitem[B]{Blander} Blander, Benjamin, {\it Local Projective
    Model Structures on Simplicial Presheaves}. K-Theory, {\bf 24}
  (2001) 283--301.

    % \bibitem[DI]{DI} Dugger, D.; Isaksen, D., {\it The motivic Adams
    % spectral sequence}.

    % \bibitem[DI-2005]{DI-Cell} Dugger, D.; Isaksen, D., {\it Motivic
    % cell structures.} Alg. Geom. Topol. {\bf 5} (2005), 615--652.

    \bibitem[DHI]{DHI} Dugger, Dan; Hollander, Sharon; Isaksen, Dan,
  {\it Hypercovers and simplicial presheaves}.

    \bibitem[DL{\O}RV]{Nordfjordeid} Dundas, B.; Levine, M.;
  {\O}stv{\ae}r, P.; R\"ondigs, O.; Voevodsky, V., {\it Motivic
    Homotopy Theory}. Springer (2000).

  %   \bibitem[FW]{Fu-Wilson} Fu, K.; Wilson, G., {\it Algorithms
  %   for computing the cohomology of the motivic Steenrod algebra}, in
  % preparation.

  %   \bibitem[G]{Geisser} Geisser, T., {\it Motivic cohomology over
  %   Dedekind rings.} (2004).

  %   \bibitem[G]{Geisser-Handbook} Geisser, T., {\it Motivic
  %   Cohomology, K-Theory, and Topological Cyclic Homology.} Handbook
  % of $K$-Theory, (2005).

  %   \bibitem[GI]{EtaLocal} Guillou, B.; Isaksen, D., {\it The
  %   $\eta$-local motivic sphere}. preprint (2014).

    \bibitem[Hir]{Hhorn} Phillip, Hirschhorn, {\it Model Categories
    and Their Localization}. AMS (2003).

    \bibitem[H-Mod]{H-Mod} Hovey, Mark, {\it Model Categories}.
  online preprint (1991).
  
    \bibitem[H-Spt]{H-Spectra} Hovey, Mark, {\it Spectra and symmetric
    spectra in general model categories}. journal? (2001).

  %   \bibitem[HKO]{HKO} Hu, P.; Kriz, I.; Ormsby, K., {\it Remarks on
  %   motivic homotopy theory over algebraically closed fields} (2008).

  %   \bibitem[HK{\O}]{HKOst} Hoyois, Marc; Kelly, Shane; {\O}stv{\ae}r,
  % Paul Arne, {\it The motivic Steenrod algebra in positive
  %   characteristic}. preprint (2013).

  %   \bibitem[HW]{BK} Haesemeyer, Christian; Weibel, Charles. {\it The
  %   Norm Residue Theorem in Motivic Cohomology}, preprint (2014).

    \bibitem[J]{Jardine} Jardine, J.F., {\it Simplicial
    presheaves}. Journal of Pure and Applied Algebra, {\bf 47} (1987)
  35--87.

    % \bibitem[L]{Levine} Levine, M., {\it A comparison of motivic and
    % classical stable homotopy theories}. arXiv:1201.0283v4, (2013).

    % \bibitem[Mc]{McCleary} McCleary, John, {\it A User's Guide to
    % Spectral Sequences}.

    \bibitem[Mor03]{Mor03} Morel, Fabien, {\it An introduction to $\bbA^1$
    homotopy theory}.

    \bibitem[Mor04]{Mor04} Morel, Fabien, {\it On the motivic $\pi_0$
    of the sphere spectrum}. NATO science series.

    \bibitem[Mor05]{Mor05} Morel, Fabien, {\it The stable $\bbA^1$
    connectivity theorems}. preprint (2004).

    % \bibitem[M-2004]{Morel04} Morel, Fabien, {\it Sur les puissances
    % de l'id\'eal fondamental de l'anneau de Witt}.

    % \bibitem[O{\O}-1]{LowDimFields} Ormsby, Kyle; Ostvaer, Paul, {\it
    % Stable Motivic $\pi_1$ of low-dimensional
    % fields}. arxiv:1310.2970v1 (2013).

  %   \bibitem[O{\O}-2]{MotBPInvQ} Ormsby, Kyle; Ostvaer, Paul, {\it
  %   Motivic Brown-Peterson invariants of the rationals}. arxiv
  % 1208.5007v2 (2013).

  %   \bibitem[O]{MotInvPadic} Ormsby, Kyle, {\it Motivic invariants of
  %   p-adic fields}. (2010).

  %   \bibitem[P]{Pelaez} Pelaez, Pablo, {\it Multiplicative Properties
  %   of the Slice Filtration}.  December 2008.

  %   \bibitem[R]{Ravenel} Ravenel, D., {\it Complex Cobordism and
  %   Stable Homotopy Groups of Spheres.} 2nd Edition, AMS (2004).

  %   \bibitem[R{\O}]{ROst} R{\"o}ndigs, O.; {\O}stv{\ae}r, P., {\it
  %   Rigidity in motivic homotopy theory.} preprint? (2007).

  %   \bibitem[So]{Soule} Soul{\'e}, C., {\it $K$-Th{\'e}orie des
  %   anneaux d'entiers de corps de nombres et cohomologie {\'e}tale.}
  % Inventiones math. (1979).

  %   \bibitem[Sp]{Spitzweck} Spitzweck, Markus, {\it A commutative
  %   $\bbP^1$-spectrum representing motivic cohomology over Dedekind
  %   domains}. preprint, (2013).

  %   \bibitem[V]{Voev} Voevodsky, Vladimir., {\it The Milnor
  %   Conjecture}. December 1996. 

    \bibitem[Voev98]{Voev98} Voevodsky, Vladimir. {\it $\bbA^1$-Homotopy
    Theory}. Doc. Math. J., (1998) pp. 579--604.

\end{thebibliography}
\end{document}
