\documentclass{amsart}%options: 11pt, A4paper, 

\usepackage{amssymb,amsfonts,amsmath,amsthm} 
\usepackage[left=1.5truein,right=1.5truein]{geometry}%options:
\usepackage{pstricks,xkeyval,pst-xkey,pst-3dplot,pstricks-add}
\usepackage{longtable}

\newcommand{\twocomp}{{}^{{\kern -.5pt}\wedge}_2}
\newcommand{\ellcomp}{{}^{{\kern -.5pt}\wedge}_{\ell}}
\usepackage{bbm}
\newcommand{\unit}{\mathbbm{1}}
\newcommand{\SH}{\mathcal{SH}}
\newcommand{\Sm}{\mathrm{Sm}}
\newcommand{\Pshv}{\mathrm{Pshv}}
\newcommand{\Shv}{\mathrm{Shv}}
\newcommand{\Spc}{\mathrm{Spc}}
\newcommand{\Spt}{\mathrm{Spt}}
\newcommand{\Ev}{\mathrm{Ev}}
\newcommand{\Frac}{\mathrm{Frac}\,}
\newcommand{\EC}{\mathfrak{EC}}



%%%%%%%     PACKAGES      %%%%%%% 
\usepackage{colortbl}
\usepackage{enumerate}
\usepackage{graphicx}
\usepackage{mathrsfs}
%\usepackage{verse}
%\usepackage{yfonts}

%%%%%%%  XY PIC COMMANDS  %%%%%%%

\usepackage{xy} 
\xyoption{all} 
\newdir{ >}{{}*!/-12pt/@{>}} 
\newdir{ -}{{}*!/-5pt/@{}} 
\newdir{- }{{}*!/-5pt/@{}} 
\newdir{-  }{{}*!/-10pt/@{}} 
\newdir{ ^(}{{}*!/-5pt/@^{(}} 
\newdir{ _(}{{}*!/-5pt/@_{(}}
\newdir{ |}{{}*!/-5pt/@{|}}
\newdir{  |}{{}*!/-10pt/@{|}}
\newdir{_(}{{}*!/0pt/@_{(}}
\xyoption{arc}
\xyoption{ps}

%%%%%%%  PS TRICKS         %%%%%%%
\usepackage{pstricks}

%%%%%%%  AMS THEOREM TAGS  %%%%%%%

\theoremstyle{remark} 
\newtheorem*{prf}{Proof} 
\newtheorem*{beweis}{Beweis}
\newtheorem{remark}{Remark} 
\newtheorem*{remarks}{Remarks} 
\newtheorem*{solution}{Solution} 
\newtheorem{Bemerkung}{Bemerkung}
\newtheorem{observation}{Observation}

\newtheorem*{losung}{L"osung}
\newtheorem*{lösung}{Lösung}

\theoremstyle{definition} 
\newtheorem{proposition}{Proposition}[section] 
\newtheorem{definition}[proposition]{Definition} 
\newtheorem{theorem}[proposition]{Theorem}
\newtheorem{example}[proposition]{Example} 
\newtheorem{notation}{Notation}
\newtheorem{Aufgabe}{Aufgabe}
\newtheorem{convention}{Convention}
\newtheorem{corollary}[proposition]{Corollary} 
\newtheorem{lemma}[proposition]{Lemma} 

\theoremstyle{plain} 
\newtheorem{hilfssatz}{Hilfssatz}
\newtheorem{conjecture}{Conjecture} 
\newtheorem{question}{Question} 

\newtheorem{algorithm}{Algorithm} 
\newtheorem*{exercise}{Exercise} 
\newtheorem{aufgabe}{Aufgabe} 
\newtheorem{prop}{Proposition}

%%%%%%%  MY MATH OPERATORS  %%%%%%%

	%% Set Theoretic / Functions
\DeclareMathOperator{\im}{im} 		
		%% Image
\DeclareMathOperator{\id}{id} 
		%% identity map 
\DeclareMathOperator{\card}{card} 
		%% Cardinality of a set 

	%% Algebraic Operators
\DeclareMathOperator{\Irr}{Irr} 
		% Subalgebra of irreducible elts
\DeclareMathOperator{\Red}{Red} 
		% Subalgebra of reducible elts
\DeclareMathOperator{\Hom}{Hom} 
		% Collection of homomorphisms 
\DeclareMathOperator{\SL}{SL} 
		% Special Linear group
\DeclareMathOperator{\GL}{GL} 
		% General Linear group
\DeclareMathOperator{\wt}{wt} 
		% weight
\DeclareMathOperator{\Symb}{Symb}
		% Symbolic algebra
\DeclareMathOperator{\Span}{span}
		% Vector space spanning set
\DeclareMathOperator{\Gal}{Gal}
		% Galois group
\DeclareMathOperator{\Char}{char}
		% Characteristic of a Ring

\DeclareMathOperator{\Vect}{\underline{Vect}}
\DeclareMathOperator{\Aut}{Aut}
\DeclareMathOperator{\End}{End}
		% Automorphism Group
\DeclareMathOperator{\lcm}{lcm}
		% Least Common Multiple
\DeclareMathOperator{\mult}{mult}
		% Multiplicity
\DeclareMathOperator{\Tor}{Tor}
\DeclareMathOperator{\tor}{Tor}
		% Homology of right derived functor of tensor product
\DeclareMathOperator{\trdeg}{tr\ deg}
		% Transcendence Degree
\DeclareMathOperator{\krdim}{kr\ dim}
		% Krull Dimension
\DeclareMathOperator{\amdim}{AM\ dim}
		% Atiyah Macdonald dimension
\DeclareMathOperator{\height}{ht}
		% height of prime ideal
\DeclareMathOperator{\amd}{AM\ d}
		% Atiyah Macdonald degree
\newcommand{\amdel}{\text{AM }\delta}
		% Atiyah Macdonald delta
\DeclareMathOperator{\ord}{ord}
		% Order
\DeclareMathOperator{\Open}{Open}
		% The Open functor.
\DeclareMathOperator{\Ab}{\underline{Ab}}
		% Category of Abelian Groups
\DeclareMathOperator{\Mod}{\underline{Mod}}
		% The category of Modules (left, right, or commutative)		
\DeclareMathOperator{\Set}{\underline{Set}}
\DeclareMathOperator{\Man}{\underline{Man}}
\DeclareMathOperator{\Haus}{\underline{Haus}}
		% Set
\DeclareMathOperator{\Gp}{\underline{Gp}}
\DeclareMathOperator{\Rng}{\underline{Rng}}
\DeclareMathOperator{\Top}{\underline{Top}}
\DeclareMathOperator{\LeftMod}{{\phantom{\Mod}\hspace{-20pt}}_{Λ}{\kern -2pt}\Mod}
		% left mods
\DeclareMathOperator{\LeftModPrime}{{\phantom{\Mod}\hspace{-20pt}}_{\Lambda'}{\kern -2pt}\Mod}
\DeclareMathOperator{\Cat}{\underline{Cat}}

\DeclareMathOperator{\Ext}{Ext}
		% The right derived functor of $\Hom$
\DeclareMathOperator{\DirectLim}{\underrightarrow{\lim}}
\DeclareMathOperator{\directlim}{\underrightarrow{\lim}}
\DeclareMathOperator{\colim}{colim}
\DeclareMathOperator{\coker}{coker}
\DeclareMathOperator{\cov}{cov}
\DeclareMathOperator{\gp}{gp}
\DeclareMathOperator{\dom}{dom}
\DeclareMathOperator{\cod}{cod}
\DeclareMathOperator{\sgn}{sgn}
\DeclareMathOperator{\Tot}{Tot}
\DeclareMathOperator{\holim}{holim}
\DeclareMathOperator{\hocolim}{hocolim}

	%%Topology Operators
\DeclareMathOperator{\ofK}{\mathbf{ofK}} 
		% offene Kern
\DeclareMathOperator{\abH}{\mathbf{abH}} 
		% abgeschlossene H"ulle
\DeclareMathOperator{\Ran}{\mathbf{Ran}}
		% Rand
\DeclareMathOperator{\Int}{\mathbf{Int}}
		% Interior
\DeclareMathOperator{\Cls}{\mathbf{Cls}}
		% Closure
\DeclareMathOperator{\Bdy}{\mathbf{Bdy}}
		% Boundary
\DeclareMathOperator{\Isom}{Isom}
\DeclareMathOperator{\Transl}{Transl}
\DeclareMathOperator{\supp}{supp}
\DeclareMathOperator{\rk}{rk}
\DeclareMathOperator{\Ann}{Ann}
\DeclareMathOperator{\length}{length}
\DeclareMathOperator{\Alg}{\underline{Alg}}

%%%%%%%  NEW COMMANDS  %%%%%%%
\DeclareMathOperator{\arcsec}{arcsec}
\DeclareMathOperator{\arccot}{arccot}
\DeclareMathOperator{\arccsc}{arccsc}
\newcommand{\st}{ \, \vert \, }
\newcommand{\paren}[2]{ \left( #1 \right) } 
\newcommand{\surjectivearrow}{\twoheadrightarrow}
\newcommand{\del}{\partial}
%\renewcommand{\cal}[1]{\mathcal{#1}}

\DeclareMathOperator{\Fr}{\underline{Frö}}
\DeclareMathOperator{\SmMan}{\underline{SmMan}}
\DeclareMathOperator{\AnMan}{\calA\underline{Man}}
\DeclareMathOperator{\TopPair}{\underline{TopPair}}
\DeclareMathOperator{\CGTop}{\underline{CGTop}}
\DeclareMathOperator{\CGHaus}{\underline{CGHaus}}
\DeclareMathOperator{\CGWH}{\underline{CGWH}}
\DeclareMathOperator{\CW}{\underline{CW}}
\DeclareMathOperator{\hTop}{\underline{hTop}}
\DeclareMathOperator{\Sp}{\underline{Sp}}
\DeclareMathOperator{\Sq}{Sq}
\DeclareMathOperator{\DirectLimit}{\underrightarrow{\lim}}
\DeclareMathOperator{\InverseLimit}{\underleftarrow{\lim}}
\DeclareMathOperator{\sSet}{\underline{sSet}}
\DeclareMathOperator{\Spec}{Spec} 

%%%%%%%  FONT ABBREvS  %%%%%%%

\newcommand{\bbCP}{\mathbb{CP}} 
\newcommand{\bbRP}{\mathbb{RP}} 

\newcommand{\bbA}{\mathbb{A}} 
\newcommand{\bbB}{\mathbb{B}}
\newcommand{\bbC}{\mathbb{C}} 
\newcommand{\bbD}{\mathbb{D}}
\newcommand{\bbE}{\mathbb{E}}
\newcommand{\bbF}{\mathbb{F}} 
\newcommand{\bbG}{\mathbb{G}}
\newcommand{\bbH}{\mathbb{H}}
\newcommand{\bbI}{\mathbb{I}} 
\newcommand{\bbJ}{\mathbb{J}}
\newcommand{\bbK}{\mathbb{K}}
\newcommand{\bbL}{\mathbb{L}} 
\newcommand{\bbM}{\mathbb{M}}
\newcommand{\bbN}{\mathbb{N}}
\newcommand{\bbO}{\mathbb{O}} 
\newcommand{\bbP}{\mathbb{P}}
\newcommand{\bbQ}{\mathbb{Q}}
\newcommand{\bbR}{\mathbb{R}} 
\newcommand{\bbS}{\mathbb{S}}
\newcommand{\bbT}{\mathbb{T}}
\newcommand{\bbU}{\mathbb{U}} 
\newcommand{\bbV}{\mathbb{V}}
\newcommand{\bbW}{\mathbb{W}} 
\newcommand{\bbX}{\mathbb{X}}
\newcommand{\bbY}{\mathbb{Y}}
\newcommand{\bbZ}{\mathbb{Z}}

\newcommand{\calA}{\mathcal{A}} 
\newcommand{\calB}{\mathcal{B}}
\newcommand{\calC}{\mathcal{C}} 
\newcommand{\calD}{\mathcal{D}}
\newcommand{\calE}{\mathcal{E}}
\newcommand{\calF}{\mathcal{F}} 
\newcommand{\calG}{\mathcal{G}}
\newcommand{\calH}{\mathcal{H}}
\newcommand{\calI}{\mathcal{I}} 
\newcommand{\calJ}{\mathcal{J}}
\newcommand{\calK}{\mathcal{K}}
\newcommand{\calL}{\mathcal{L}} 
\newcommand{\calM}{\mathcal{M}}
\newcommand{\calN}{\mathcal{N}}
\newcommand{\calO}{\mathcal{O}} 
\newcommand{\calP}{\mathcal{P}}
\newcommand{\calQ}{\mathcal{Q}}
\newcommand{\calR}{\mathcal{R}} 
\newcommand{\calS}{\mathcal{S}}
\newcommand{\calT}{\mathcal{T}}
\newcommand{\calU}{\mathcal{U}} 
\newcommand{\calV}{\mathcal{V}}
\newcommand{\calW}{\mathcal{W}} 
\newcommand{\calX}{\mathcal{X}}
\newcommand{\calY}{\mathcal{Y}}
\newcommand{\calZ}{\mathcal{Z}}

\newcommand{\cala}{\mathcal{a}} 
\newcommand{\calb}{\mathcal{b}}
\newcommand{\calc}{\mathcal{c}} 
\newcommand{\cald}{\mathcal{d}}
\newcommand{\cale}{\mathcal{e}}
\newcommand{\calf}{\mathcal{f}} 
\newcommand{\calg}{\mathcal{g}}
\newcommand{\calh}{\mathcal{h}}
\newcommand{\cali}{\mathcal{i}} 
\newcommand{\calj}{\mathcal{j}}
\newcommand{\calk}{\mathcal{k}}
\newcommand{\call}{\ell} 
\newcommand{\calm}{\mathcal{m}}
\newcommand{\caln}{\mathcal{n}}
\newcommand{\calo}{\mathcal{o}} 
\newcommand{\calp}{\mathcal{p}}
\newcommand{\calq}{\mathcal{q}}
\newcommand{\calr}{\mathcal{r}} 
\newcommand{\cals}{\mathcal{s}}
\newcommand{\calt}{\mathcal{t}}
\newcommand{\calu}{\mathcal{u}} 
\newcommand{\calv}{\mathcal{v}}
\newcommand{\calw}{\mathcal{w}} 
\newcommand{\calx}{\mathcal{x}}
\newcommand{\caly}{\mathcal{y}}
\newcommand{\calz}{\mathcal{z}}

\newcommand{\frakA}{\mathfrak{A}} 
\newcommand{\frakB}{\mathfrak{B}}
\newcommand{\frakC}{\mathfrak{C}} 
\newcommand{\frakD}{\mathfrak{D}}
\newcommand{\frakE}{\mathfrak{E}}
\newcommand{\frakF}{\mathfrak{F}} 
\newcommand{\frakG}{\mathfrak{G}}
\newcommand{\frakH}{\mathfrak{H}}
\newcommand{\frakI}{\mathfrak{I}} 
\newcommand{\frakJ}{\mathfrak{J}}
\newcommand{\frakK}{\mathfrak{K}}
\newcommand{\frakL}{\mathfrak{L}} 
\newcommand{\frakM}{\mathfrak{M}}
\newcommand{\frakN}{\mathfrak{N}}
\newcommand{\frakO}{\mathfrak{O}} 
\newcommand{\frakP}{\mathfrak{P}}
\newcommand{\frakQ}{\mathfrak{Q}}
\newcommand{\frakR}{\mathfrak{R}} 
\newcommand{\frakS}{\mathfrak{S}}
\newcommand{\frakT}{\mathfrak{T}}
\newcommand{\frakU}{\mathfrak{U}} 
\newcommand{\frakV}{\mathfrak{V}}
\newcommand{\frakW}{\mathfrak{W}} 
\newcommand{\frakX}{\mathfrak{X}}
\newcommand{\frakY}{\mathfrak{Y}}
\newcommand{\frakZ}{\mathfrak{Z}}

\newcommand{\fraka}{\mathfrak{a}} 
\newcommand{\frakb}{\mathfrak{b}}
\newcommand{\frakc}{\mathfrak{c}} 
\newcommand{\frakd}{\mathfrak{d}}
\newcommand{\frake}{\mathfrak{e}}
\newcommand{\frakf}{\mathfrak{f}} 
\newcommand{\frakg}{\mathfrak{g}}
\newcommand{\frakh}{\mathfrak{h}}
\newcommand{\fraki}{\mathfrak{i}} 
\newcommand{\frakj}{\mathfrak{j}}
\newcommand{\frakk}{\mathfrak{k}}
\newcommand{\frakl}{\mathfrak{l}} 
\newcommand{\frakm}{\mathfrak{m}}
\newcommand{\frakn}{\mathfrak{n}}
\newcommand{\frako}{\mathfrak{o}} 
\newcommand{\frakp}{\mathfrak{p}}
\newcommand{\frakq}{\mathfrak{q}}
\newcommand{\frakr}{\mathfrak{r}} 
\newcommand{\fraks}{\mathfrak{s}}
\newcommand{\frakt}{\mathfrak{t}}
\newcommand{\fraku}{\mathfrak{u}} 
\newcommand{\frakv}{\mathfrak{v}}
\newcommand{\frakw}{\mathfrak{w}} 
\newcommand{\frakx}{\mathfrak{x}}
\newcommand{\fraky}{\mathfrak{y}}
\newcommand{\frakz}{\mathfrak{z}}

\newcommand{\rmA}{\textrm{A}} 
\newcommand{\rmB}{\textrm{B}}
\newcommand{\rmC}{\textrm{C}} 
\newcommand{\rmD}{\textrm{D}}
\newcommand{\rmE}{\textrm{E}}
\newcommand{\rmF}{\textrm{F}} 
\newcommand{\rmG}{\textrm{G}}
\newcommand{\rmH}{\textrm{H}}
\newcommand{\rmI}{\textrm{I}} 
\newcommand{\rmJ}{\textrm{J}}
\newcommand{\rmK}{\textrm{K}}
\newcommand{\rmL}{\textrm{L}} 
\newcommand{\rmM}{\textrm{M}}
\newcommand{\rmN}{\textrm{N}}
\newcommand{\rmO}{\textrm{O}} 
\newcommand{\rmP}{\textrm{P}}
\newcommand{\rmQ}{\textrm{Q}}
\newcommand{\rmR}{\textrm{R}} 
\newcommand{\rmS}{\textrm{S}}
\newcommand{\rmT}{\textrm{T}}
\newcommand{\rmU}{\textrm{U}} 
\newcommand{\rmV}{\textrm{V}}
\newcommand{\rmW}{\textrm{W}} 
\newcommand{\rmX}{\textrm{X}}
\newcommand{\rmY}{\textrm{Y}}
\newcommand{\rmZ}{\textrm{Z}}

\newcommand{\Mu}{\textrm{M}}
\newcommand{\Tau}{\textrm{T}}

\newcommand{\scrA}{\mathscr{A}} 
\newcommand{\scrB}{\mathscr{B}}
\newcommand{\scrC}{\mathscr{C}} 
\newcommand{\scrD}{\mathscr{D}}
\newcommand{\scrE}{\mathscr{E}}
\newcommand{\scrF}{\mathscr{F}} 
\newcommand{\scrG}{\mathscr{G}}
\newcommand{\scrH}{\mathscr{H}}
\newcommand{\scrI}{\mathscr{I}} 
\newcommand{\scrJ}{\mathscr{J}}
\newcommand{\scrK}{\mathscr{K}}
\newcommand{\scrL}{\mathscr{L}} 
\newcommand{\scrM}{\mathscr{M}}
\newcommand{\scrN}{\mathscr{N}}
\newcommand{\scrO}{\mathscr{O}} 
\newcommand{\scrP}{\mathscr{P}}
\newcommand{\scrQ}{\mathscr{Q}}
\newcommand{\scrR}{\mathscr{R}} 
\newcommand{\scrS}{\mathscr{S}}
\newcommand{\scrT}{\mathscr{T}}
\newcommand{\scrU}{\mathscr{U}} 
\newcommand{\scrV}{\mathscr{V}}
\newcommand{\scrW}{\mathscr{W}} 
\newcommand{\scrX}{\mathscr{X}}
\newcommand{\scrY}{\mathscr{Y}}
\newcommand{\scrZ}{\mathscr{Z}}

%%%%%%%%Spacing%%%%%%%%%%%%%%%%%%%
\newcommand{\tab}{\hspace{3ex}}


\newcommand{\bispt}{\Spt_{S^1,\bbG_m}}
\newcommand{\symspt}{\Spt^{\Sigma}}
\newcommand{\Fpbar}{\overline{\bbF}_p}

\begin{document}

\section{Voevodsky's connectivity theorem for $\bbP^1$-spectra}

The following is from an email from Aravind to me concerning the proof
of the connectivity theorem for $\bbP^1$ spectra.
\begin{quotation}
  Most of the proof appears in Morel's Trieste notes (in the section
  on the homotopy t-structure for $P^1$-spectra).  First you prove a
  connectivity result for $P^1$-stable homotopy sheaves by using
  Morel's $S^1$-stable connectivity theorem and studying what happens
  under $G_m$ loops and $G_m$-suspension: suspension preserves
  connectivity, and Morel shows that taking $G_m$-loops has the effect
  of making a ``contraction''.  Then, you globalize this.
\end{quotation}

Our goal is to prove theorem 4.14 of \cite{Voev98}, which should be
restated in terms of $\bbP^1$-spectra. 

\begin{theorem}
  Let $(X,x)$ be a pointed smooth scheme over $\Spec(k)$ where $k$ is
  an infinite field. Let $(Y,y)$ be a pointed simplicial sheaf. Then
  for any $n< \dim(X)$ 
  \begin{equation*}
    \SH(k)(S^{n}\wedge \bbG_m^{m}\wedge \Sigma^{\infty}X, \Sigma^{\infty} Y) = 0.
  \end{equation*}

  In particular, if we take $X = S^0$, this theorem says 
  \begin{equation*}
    \SH(k)(S^n\wedge\bbG_m^m\wedge \unit, \Sigma^{\infty}Y) = \tilde{\pi}_n(Y)_{-m} = 0
  \end{equation*}
  whenever $n<0$. 
\end{theorem}

We can formulate the theorem by using homotopy sheaves. The theorem
says (in Morel's notation) that
$\tilde{\pi}_{n}(\Sigma^{\infty}Y)_m(X)$ vanishes whenever
$n < \dim(X)$.

The theorem is equivalent to the vanishing of the homotopy groups
$\pi_{n}(\Sigma^{\infty}Y)_m(X)$ when $n < \dim(X)$. (Is this right?)
So certainly by \cite[Example 5.2.2]{Mor03} we have
$\pi_n(\Sigma^{\infty}Y)_m = 0$ as a sheaf whenever $n < 0$. How to
show $\pi_n(\Sigma^{\infty}Y)_m(X) = 0$ when $0\leq n < \dim X$?


\section{Assumptions from previous lectures}

\subsection{Facts about Nisnevich topology}

\begin{proposition}\cite[2.4.1]{Mor04}
  Let $M$ be a sheaf of abelian groups on $\Sm/k$, and let
  $X\in \Sm/k$ with Krull dimension $d$. Then whenever $n > d$,
  $H^n_{Nis}(X;M) = 0$.
\end{proposition}

\begin{proposition}\cite[2.4.1]{Mor04}
  For any $X \in \Sm/k$, and for any $x \in X(k)$, there is an
  isomorphism of pointed sheaves of sets in the Nisnevich topology
  \begin{equation*}
    X/(X-\{x\}) \cong \bbA^n / (\bbA^n - \{0\}).
  \end{equation*}
\end{proposition}

\subsection{$S^1$-spectra}

\begin{definition}
  Let $\SH_s^{S^1}(k)$ denote the homotopy category associated to the
  projective model structure on Nisnevich sheaves of simplicial
  $S^1$-spectra on $\Sm/k$. (Is this the same as localizing the
  collection of $S^1$ spectra of spaces equipped with the proj. model
  str?)
\end{definition}

\begin{definition}
  Let $\SH^{S^1}(k)$ denote the localization of the model category
  associated to $\SH_s^{S^1}(k)$ at the collection of maps
  $E \wedge \bbA^1 \to E$.
\end{definition}

\begin{remark}
  Advantages to using alternate model category structures? E.g., use
  presheaves instead of sheaves? Use Hovey's method of stabilizing wrt
  Quillen functor $X \wedge -$? 
\end{remark}

\begin{definition}
  An $S^1$-spectrum $E$ is said to be $n$-connected if for any
  $m\leq n$, the homotopy sheaves $\pi_m(E)$ are trivial. 
\end{definition}

\begin{definition}
  Let $\frakC$ be a triangulated category. A $t$-structure on $\frakC$
  is a pair of full subcategories $(\frakC_{\geq 0}, \frakC_{\leq 0})$
  which satisfies 
  \begin{enumerate}
      \item
    $(\forall X \in \frakC_{\geq 0}) (\forall Y \in \frakC_{\leq 0})(
    \Hom_{\frakC}(X, Y[-1])=0$
      \item $\frakC_{\geq 0}[1] \subseteq \frakC_{\geq 0}$ and
    $\frakC_{\leq 0}[-1] \subseteq \frakC_{\leq 0}$ 

      \item for any $X \in \frakC$ there exists a distinguished triangle 
    \begin{equation*}
      Y \to X \to Z \to Y[1]
    \end{equation*}
    for which $Y \in \frakC_{\geq 0}$, $Z\in \frakC_{\leq 0}[-1]$..
  \end{enumerate}

  The heart of a $t$-structure is the full subcategory given by
  $\frakC_{\geq 0} \cap \frakC_{\leq 0}$. 
\end{definition}


\begin{definition}[$t$-structure on $\SH_s^{S^1}(k)$]
  Define $\SH_s^{S^1}(k)_{\geq 0}$ to be the full subcategory of
  $\SH_s^{S^1}(k)$ consisting of objects $E$ such that $\pi_n(E)=0$
  whenver $n<0$.
  
  Define $\SH_s^{S^1}(k)_{\leq 0}$ to be the full subcategory of
  $\SH_s^{S^1}(k)$ consisting of objects $E$ such that $\pi_n(E)=0$
  whenver $n>0$.
\end{definition}

\begin{theorem}
  The triple
  $(\SH_s^{S^1}(k), \SH_s^{S^1}(k)_{\geq 0}, \SH_s^{S^1}(k)_{\leq 0})$
  is a $t$-structure on $\SH_s^{S^1}(k)$. 
\end{theorem}

% \begin{definition}
%   An $S^1$-spectrum $F$ is $\bbA^1$
% \end{definition}

\begin{proposition}\cite[Lemma4.2.4]{Mor03}
  The functor
  $L^{\infty} : \Spt^{S^1}_s(k) \to \Spt^{S^1}_{s,\bbA^1}(k)$
  identifies the $\bbA^1$-localized $S^1$ stable homotopy category
  with the homotopy category of $\bbA^1$-local $S^1$ spectra.
\end{proposition}

\begin{theorem}[$S^1$ stable connectivity theorem]
  Let $E \in \SH_s^{S^1}(k)$, and suppose that whenever $n < 0$ the
  sheaf $\pi_n E = 0$. Then for all $n<0$, $\pi_n L_{\bbA^1}E = 0$.
\end{theorem}

\begin{theorem}
  The pair $(\SH_{\geq 0}^{S^1}(k), \SH_{\leq 0}^{S^1}(k))$ is a
  $t$-structure on the category $\SH^{S^1}(k)$. 
\end{theorem}

\begin{definition}
  Strictly $\bbA^1$ invariant sheaf of Abelian groups.

  If $M$ is strictly $\bbA^1$ invariant sheaf of groups, define the
  Eilenberg-MacLane spectrum $HM$ associated to it. 
\end{definition}

\begin{proposition}
  $HM$ is $\bbA^1$ local iff $M$ is strictly $\bbA^1$ invariant.
\end{proposition}

\begin{proposition}
  The heart of the homotopy $t$ structure is equivalent to the
  category of strictly $\bbA^1$ invariant sheaves. 
\end{proposition}



\section{Inverting $\bbG_m$; $\bbP^1$ spectra}

\subsection{$\bbG_m$ suspension and loops}

\subsection{Homotopy sheaves of $\underline{Hom}(\bbG_m, E)$}

\begin{definition}
  Contraction of a sheaf of pointed sets $G$ (or abelian groups $G$). $G_{-1}$
\end{definition}

\begin{theorem}\cite[Lemma 4.3.11]{Mor03}
  $\pi_n(\underline{Hom}(\bbG_m, E)) \to \pi_{n}(E)_{-1}$ is iso. 
\end{theorem}

\begin{lemma}
  If $M$ is a strictly $\bbA^1$ invariant sheaf of abelian groups, then
  \begin{equation*}
    \underline{Hom}(\bbG_m, HM) \cong H(M_{-1})
  \end{equation*}
\end{lemma}

\begin{lemma}
When $n\neq 0$,
  \begin{equation}
    [\Sigma^{\infty}\bbG_m, HM[n]]_s^{S^1} = 0.
  \end{equation}

The following map is an iso. 
  \begin{equation}
    [\Sigma^{\infty}\bbG_m, HM]_s^{S^1} \to [S^0, H(M_{-1})]_s^{S^1}
  \end{equation}
\end{lemma}

\begin{thebibliography}{OOOO}
    \bibitem[A-1974]{Adams} Adams, J.F., {\it Stable Homotopy and
    Generalized Homology.} Chicago Lectures in Mathematics, (1974).

    % \bibitem[A-1965]{Adams-Periodicity} Adams, J.F., {\it A
    % periodicity theorem in homological
    % algebra}. Proc. Camb. Phil. Soc. (1966).

    \bibitem[B]{Blander} Blander, Benjamin, {\it Local Projective
    Model Structures on Simplicial Presheaves}. K-Theory, {\bf 24}
  (2001) 283--301.

    % \bibitem[DI]{DI} Dugger, D.; Isaksen, D., {\it The motivic Adams
    % spectral sequence}.

    % \bibitem[DI-2005]{DI-Cell} Dugger, D.; Isaksen, D., {\it Motivic
    % cell structures.} Alg. Geom. Topol. {\bf 5} (2005), 615--652.

    \bibitem[DHI]{DHI} Dugger, Dan; Hollander, Sharon; Isaksen, Dan,
  {\it Hypercovers and simplicial presheaves}.

    \bibitem[DL{\O}RV]{Nordfjordeid} Dundas, B.; Levine, M.;
  {\O}stv{\ae}r, P.; R\"ondigs, O.; Voevodsky, V., {\it Motivic
    Homotopy Theory}. Springer (2000).

  %   \bibitem[FW]{Fu-Wilson} Fu, K.; Wilson, G., {\it Algorithms
  %   for computing the cohomology of the motivic Steenrod algebra}, in
  % preparation.

  %   \bibitem[G]{Geisser} Geisser, T., {\it Motivic cohomology over
  %   Dedekind rings.} (2004).

  %   \bibitem[G]{Geisser-Handbook} Geisser, T., {\it Motivic
  %   Cohomology, K-Theory, and Topological Cyclic Homology.} Handbook
  % of $K$-Theory, (2005).

  %   \bibitem[GI]{EtaLocal} Guillou, B.; Isaksen, D., {\it The
  %   $\eta$-local motivic sphere}. preprint (2014).

    \bibitem[Hir]{Hhorn} Phillip, Hirschhorn, {\it Model Categories
    and Their Localization}. AMS (2003).

    \bibitem[H-Mod]{H-Mod} Hovey, Mark, {\it Model Categories}.
  online preprint (1991).
  
    \bibitem[H-Spt]{H-Spectra} Hovey, Mark, {\it Spectra and symmetric
    spectra in general model categories}. journal? (2001).

  %   \bibitem[HKO]{HKO} Hu, P.; Kriz, I.; Ormsby, K., {\it Remarks on
  %   motivic homotopy theory over algebraically closed fields} (2008).

  %   \bibitem[HK{\O}]{HKOst} Hoyois, Marc; Kelly, Shane; {\O}stv{\ae}r,
  % Paul Arne, {\it The motivic Steenrod algebra in positive
  %   characteristic}. preprint (2013).

  %   \bibitem[HW]{BK} Haesemeyer, Christian; Weibel, Charles. {\it The
  %   Norm Residue Theorem in Motivic Cohomology}, preprint (2014).

    \bibitem[J]{Jardine} Jardine, J.F., {\it Simplicial
    presheaves}. Journal of Pure and Applied Algebra, {\bf 47} (1987)
  35--87.

    % \bibitem[L]{Levine} Levine, M., {\it A comparison of motivic and
    % classical stable homotopy theories}. arXiv:1201.0283v4, (2013).

    % \bibitem[Mc]{McCleary} McCleary, John, {\it A User's Guide to
    % Spectral Sequences}.

    \bibitem[Mor03]{Mor03} Morel, Fabien, {\it An introduction to $\bbA^1$
    homotopy theory}.

    \bibitem[Mor04]{Mor04} Morel, Fabien, {\it On the motivic $\pi_0$
    of the sphere spectrum}. NATO science series.

    \bibitem[Mor05]{Mor05} Morel, Fabien, {\it The stable $\bbA^1$
    connectivity theorems}. preprint (2004).

    % \bibitem[M-2004]{Morel04} Morel, Fabien, {\it Sur les puissances
    % de l'id\'eal fondamental de l'anneau de Witt}.

    % \bibitem[O{\O}-1]{LowDimFields} Ormsby, Kyle; Ostvaer, Paul, {\it
    % Stable Motivic $\pi_1$ of low-dimensional
    % fields}. arxiv:1310.2970v1 (2013).

  %   \bibitem[O{\O}-2]{MotBPInvQ} Ormsby, Kyle; Ostvaer, Paul, {\it
  %   Motivic Brown-Peterson invariants of the rationals}. arxiv
  % 1208.5007v2 (2013).

  %   \bibitem[O]{MotInvPadic} Ormsby, Kyle, {\it Motivic invariants of
  %   p-adic fields}. (2010).

  %   \bibitem[P]{Pelaez} Pelaez, Pablo, {\it Multiplicative Properties
  %   of the Slice Filtration}.  December 2008.

  %   \bibitem[R]{Ravenel} Ravenel, D., {\it Complex Cobordism and
  %   Stable Homotopy Groups of Spheres.} 2nd Edition, AMS (2004).

  %   \bibitem[R{\O}]{ROst} R{\"o}ndigs, O.; {\O}stv{\ae}r, P., {\it
  %   Rigidity in motivic homotopy theory.} preprint? (2007).

  %   \bibitem[So]{Soule} Soul{\'e}, C., {\it $K$-Th{\'e}orie des
  %   anneaux d'entiers de corps de nombres et cohomologie {\'e}tale.}
  % Inventiones math. (1979).

  %   \bibitem[Sp]{Spitzweck} Spitzweck, Markus, {\it A commutative
  %   $\bbP^1$-spectrum representing motivic cohomology over Dedekind
  %   domains}. preprint, (2013).

  %   \bibitem[V]{Voev} Voevodsky, Vladimir., {\it The Milnor
  %   Conjecture}. December 1996. 

    \bibitem[Voev98]{Voev98} Voevodsky, Vladimir. {\it $\bbA^1$-Homotopy
    Theory}. Doc. Math. J., (1998) pp. 579--604.

\end{thebibliography}
\end{document}
