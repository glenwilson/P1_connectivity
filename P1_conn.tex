\documentclass{amsart}%options: 11pt, A4paper, 

\usepackage{amssymb,amsfonts,amsmath,amsthm} 
\usepackage[left=1.5truein,right=1.5truein]{geometry}%options:
\usepackage{pstricks,xkeyval,pst-xkey,pst-3dplot,pstricks-add}
\usepackage{longtable}

\newcommand{\twocomp}{{}^{{\kern -.5pt}\wedge}_2}
\newcommand{\ellcomp}{{}^{{\kern -.5pt}\wedge}_{\ell}}
\usepackage{bbm}
\newcommand{\unit}{\mathbbm{1}}
\newcommand{\SH}{\mathcal{SH}}
\newcommand{\Sm}{\mathrm{Sm}}
\newcommand{\Pshv}{\mathrm{Pshv}}
\newcommand{\Shv}{\mathrm{Shv}}
\newcommand{\Spc}{\mathrm{Spc}}
\newcommand{\Spt}{\mathrm{Spt}}
\newcommand{\Ev}{\mathrm{Ev}}
\newcommand{\Frac}{\mathrm{Frac}\,}
\newcommand{\EC}{\mathfrak{EC}}



%%%%%%%     PACKAGES      %%%%%%% 
\usepackage{colortbl}
\usepackage{enumerate}
\usepackage{graphicx}
\usepackage{mathrsfs}
%\usepackage{verse}
%\usepackage{yfonts}

%%%%%%%  XY PIC COMMANDS  %%%%%%%

\usepackage{xy} 
\xyoption{all} 
\newdir{ >}{{}*!/-12pt/@{>}} 
\newdir{ -}{{}*!/-5pt/@{}} 
\newdir{- }{{}*!/-5pt/@{}} 
\newdir{-  }{{}*!/-10pt/@{}} 
\newdir{ ^(}{{}*!/-5pt/@^{(}} 
\newdir{ _(}{{}*!/-5pt/@_{(}}
\newdir{ |}{{}*!/-5pt/@{|}}
\newdir{  |}{{}*!/-10pt/@{|}}
\newdir{_(}{{}*!/0pt/@_{(}}
\xyoption{arc}
\xyoption{ps}

%%%%%%%  PS TRICKS         %%%%%%%
\usepackage{pstricks}

%%%%%%%  AMS THEOREM TAGS  %%%%%%%

\theoremstyle{remark} 
\newtheorem*{prf}{Proof} 
\newtheorem*{beweis}{Beweis}
\newtheorem{remark}{Remark} 
\newtheorem*{remarks}{Remarks} 
\newtheorem*{solution}{Solution} 
\newtheorem{Bemerkung}{Bemerkung}
\newtheorem{observation}{Observation}

\newtheorem*{losung}{L"osung}
\newtheorem*{lösung}{Lösung}

\theoremstyle{definition} 
\newtheorem{proposition}{Proposition}[section] 
\newtheorem{definition}[proposition]{Definition} 
\newtheorem{theorem}[proposition]{Theorem}
\newtheorem{example}[proposition]{Example} 
\newtheorem{notation}{Notation}
\newtheorem{Aufgabe}{Aufgabe}
\newtheorem{convention}{Convention}
\newtheorem{corollary}[proposition]{Corollary} 
\newtheorem{lemma}[proposition]{Lemma} 

\theoremstyle{plain} 
\newtheorem{hilfssatz}{Hilfssatz}
\newtheorem{conjecture}{Conjecture} 
\newtheorem{question}{Question} 

\newtheorem{algorithm}{Algorithm} 
\newtheorem*{exercise}{Exercise} 
\newtheorem{aufgabe}{Aufgabe} 
\newtheorem{prop}{Proposition}

%%%%%%%  MY MATH OPERATORS  %%%%%%%

	%% Set Theoretic / Functions
\DeclareMathOperator{\im}{im} 		
		%% Image
\DeclareMathOperator{\id}{id} 
		%% identity map 
\DeclareMathOperator{\card}{card} 
		%% Cardinality of a set 

	%% Algebraic Operators
\DeclareMathOperator{\Irr}{Irr} 
		% Subalgebra of irreducible elts
\DeclareMathOperator{\Red}{Red} 
		% Subalgebra of reducible elts
\DeclareMathOperator{\Hom}{Hom} 
		% Collection of homomorphisms 
\DeclareMathOperator{\SL}{SL} 
		% Special Linear group
\DeclareMathOperator{\GL}{GL} 
		% General Linear group
\DeclareMathOperator{\wt}{wt} 
		% weight
\DeclareMathOperator{\Symb}{Symb}
		% Symbolic algebra
\DeclareMathOperator{\Span}{span}
		% Vector space spanning set
\DeclareMathOperator{\Gal}{Gal}
		% Galois group
\DeclareMathOperator{\Char}{char}
		% Characteristic of a Ring

\DeclareMathOperator{\Vect}{\underline{Vect}}
\DeclareMathOperator{\Aut}{Aut}
\DeclareMathOperator{\End}{End}
		% Automorphism Group
\DeclareMathOperator{\lcm}{lcm}
		% Least Common Multiple
\DeclareMathOperator{\mult}{mult}
		% Multiplicity
\DeclareMathOperator{\Tor}{Tor}
\DeclareMathOperator{\tor}{Tor}
		% Homology of right derived functor of tensor product
\DeclareMathOperator{\trdeg}{tr\ deg}
		% Transcendence Degree
\DeclareMathOperator{\krdim}{kr\ dim}
		% Krull Dimension
\DeclareMathOperator{\amdim}{AM\ dim}
		% Atiyah Macdonald dimension
\DeclareMathOperator{\height}{ht}
		% height of prime ideal
\DeclareMathOperator{\amd}{AM\ d}
		% Atiyah Macdonald degree
\newcommand{\amdel}{\text{AM }\delta}
		% Atiyah Macdonald delta
\DeclareMathOperator{\ord}{ord}
		% Order
\DeclareMathOperator{\Open}{Open}
		% The Open functor.
\DeclareMathOperator{\Ab}{\underline{Ab}}
		% Category of Abelian Groups
\DeclareMathOperator{\Mod}{\underline{Mod}}
		% The category of Modules (left, right, or commutative)		
\DeclareMathOperator{\Set}{\underline{Set}}
\DeclareMathOperator{\Man}{\underline{Man}}
\DeclareMathOperator{\Haus}{\underline{Haus}}
		% Set
\DeclareMathOperator{\Gp}{\underline{Gp}}
\DeclareMathOperator{\Rng}{\underline{Rng}}
\DeclareMathOperator{\Top}{\underline{Top}}
\DeclareMathOperator{\LeftMod}{{\phantom{\Mod}\hspace{-20pt}}_{Λ}{\kern -2pt}\Mod}
		% left mods
\DeclareMathOperator{\LeftModPrime}{{\phantom{\Mod}\hspace{-20pt}}_{\Lambda'}{\kern -2pt}\Mod}
\DeclareMathOperator{\Cat}{\underline{Cat}}

\DeclareMathOperator{\Ext}{Ext}
		% The right derived functor of $\Hom$
\DeclareMathOperator{\DirectLim}{\underrightarrow{\lim}}
\DeclareMathOperator{\directlim}{\underrightarrow{\lim}}
\DeclareMathOperator{\colim}{colim}
\DeclareMathOperator{\coker}{coker}
\DeclareMathOperator{\cov}{cov}
\DeclareMathOperator{\gp}{gp}
\DeclareMathOperator{\dom}{dom}
\DeclareMathOperator{\cod}{cod}
\DeclareMathOperator{\sgn}{sgn}
\DeclareMathOperator{\Tot}{Tot}
\DeclareMathOperator{\holim}{holim}
\DeclareMathOperator{\hocolim}{hocolim}

	%%Topology Operators
\DeclareMathOperator{\ofK}{\mathbf{ofK}} 
		% offene Kern
\DeclareMathOperator{\abH}{\mathbf{abH}} 
		% abgeschlossene H"ulle
\DeclareMathOperator{\Ran}{\mathbf{Ran}}
		% Rand
\DeclareMathOperator{\Int}{\mathbf{Int}}
		% Interior
\DeclareMathOperator{\Cls}{\mathbf{Cls}}
		% Closure
\DeclareMathOperator{\Bdy}{\mathbf{Bdy}}
		% Boundary
\DeclareMathOperator{\Isom}{Isom}
\DeclareMathOperator{\Transl}{Transl}
\DeclareMathOperator{\supp}{supp}
\DeclareMathOperator{\rk}{rk}
\DeclareMathOperator{\Ann}{Ann}
\DeclareMathOperator{\length}{length}
\DeclareMathOperator{\Alg}{\underline{Alg}}

%%%%%%%  NEW COMMANDS  %%%%%%%
\DeclareMathOperator{\arcsec}{arcsec}
\DeclareMathOperator{\arccot}{arccot}
\DeclareMathOperator{\arccsc}{arccsc}
\newcommand{\st}{ \, \vert \, }
\newcommand{\paren}[2]{ \left( #1 \right) } 
\newcommand{\surjectivearrow}{\twoheadrightarrow}
\newcommand{\del}{\partial}
%\renewcommand{\cal}[1]{\mathcal{#1}}

\DeclareMathOperator{\Fr}{\underline{Frö}}
\DeclareMathOperator{\SmMan}{\underline{SmMan}}
\DeclareMathOperator{\AnMan}{\calA\underline{Man}}
\DeclareMathOperator{\TopPair}{\underline{TopPair}}
\DeclareMathOperator{\CGTop}{\underline{CGTop}}
\DeclareMathOperator{\CGHaus}{\underline{CGHaus}}
\DeclareMathOperator{\CGWH}{\underline{CGWH}}
\DeclareMathOperator{\CW}{\underline{CW}}
\DeclareMathOperator{\hTop}{\underline{hTop}}
\DeclareMathOperator{\Sp}{\underline{Sp}}
\DeclareMathOperator{\Sq}{Sq}
\DeclareMathOperator{\DirectLimit}{\underrightarrow{\lim}}
\DeclareMathOperator{\InverseLimit}{\underleftarrow{\lim}}
\DeclareMathOperator{\sSet}{\underline{sSet}}
\DeclareMathOperator{\Spec}{Spec} 

%%%%%%%  FONT ABBREvS  %%%%%%%

\newcommand{\bbCP}{\mathbb{CP}} 
\newcommand{\bbRP}{\mathbb{RP}} 

\newcommand{\bbA}{\mathbb{A}} 
\newcommand{\bbB}{\mathbb{B}}
\newcommand{\bbC}{\mathbb{C}} 
\newcommand{\bbD}{\mathbb{D}}
\newcommand{\bbE}{\mathbb{E}}
\newcommand{\bbF}{\mathbb{F}} 
\newcommand{\bbG}{\mathbb{G}}
\newcommand{\bbH}{\mathbb{H}}
\newcommand{\bbI}{\mathbb{I}} 
\newcommand{\bbJ}{\mathbb{J}}
\newcommand{\bbK}{\mathbb{K}}
\newcommand{\bbL}{\mathbb{L}} 
\newcommand{\bbM}{\mathbb{M}}
\newcommand{\bbN}{\mathbb{N}}
\newcommand{\bbO}{\mathbb{O}} 
\newcommand{\bbP}{\mathbb{P}}
\newcommand{\bbQ}{\mathbb{Q}}
\newcommand{\bbR}{\mathbb{R}} 
\newcommand{\bbS}{\mathbb{S}}
\newcommand{\bbT}{\mathbb{T}}
\newcommand{\bbU}{\mathbb{U}} 
\newcommand{\bbV}{\mathbb{V}}
\newcommand{\bbW}{\mathbb{W}} 
\newcommand{\bbX}{\mathbb{X}}
\newcommand{\bbY}{\mathbb{Y}}
\newcommand{\bbZ}{\mathbb{Z}}

\newcommand{\calA}{\mathcal{A}} 
\newcommand{\calB}{\mathcal{B}}
\newcommand{\calC}{\mathcal{C}} 
\newcommand{\calD}{\mathcal{D}}
\newcommand{\calE}{\mathcal{E}}
\newcommand{\calF}{\mathcal{F}} 
\newcommand{\calG}{\mathcal{G}}
\newcommand{\calH}{\mathcal{H}}
\newcommand{\calI}{\mathcal{I}} 
\newcommand{\calJ}{\mathcal{J}}
\newcommand{\calK}{\mathcal{K}}
\newcommand{\calL}{\mathcal{L}} 
\newcommand{\calM}{\mathcal{M}}
\newcommand{\calN}{\mathcal{N}}
\newcommand{\calO}{\mathcal{O}} 
\newcommand{\calP}{\mathcal{P}}
\newcommand{\calQ}{\mathcal{Q}}
\newcommand{\calR}{\mathcal{R}} 
\newcommand{\calS}{\mathcal{S}}
\newcommand{\calT}{\mathcal{T}}
\newcommand{\calU}{\mathcal{U}} 
\newcommand{\calV}{\mathcal{V}}
\newcommand{\calW}{\mathcal{W}} 
\newcommand{\calX}{\mathcal{X}}
\newcommand{\calY}{\mathcal{Y}}
\newcommand{\calZ}{\mathcal{Z}}

\newcommand{\cala}{\mathcal{a}} 
\newcommand{\calb}{\mathcal{b}}
\newcommand{\calc}{\mathcal{c}} 
\newcommand{\cald}{\mathcal{d}}
\newcommand{\cale}{\mathcal{e}}
\newcommand{\calf}{\mathcal{f}} 
\newcommand{\calg}{\mathcal{g}}
\newcommand{\calh}{\mathcal{h}}
\newcommand{\cali}{\mathcal{i}} 
\newcommand{\calj}{\mathcal{j}}
\newcommand{\calk}{\mathcal{k}}
\newcommand{\call}{\ell} 
\newcommand{\calm}{\mathcal{m}}
\newcommand{\caln}{\mathcal{n}}
\newcommand{\calo}{\mathcal{o}} 
\newcommand{\calp}{\mathcal{p}}
\newcommand{\calq}{\mathcal{q}}
\newcommand{\calr}{\mathcal{r}} 
\newcommand{\cals}{\mathcal{s}}
\newcommand{\calt}{\mathcal{t}}
\newcommand{\calu}{\mathcal{u}} 
\newcommand{\calv}{\mathcal{v}}
\newcommand{\calw}{\mathcal{w}} 
\newcommand{\calx}{\mathcal{x}}
\newcommand{\caly}{\mathcal{y}}
\newcommand{\calz}{\mathcal{z}}

\newcommand{\frakA}{\mathfrak{A}} 
\newcommand{\frakB}{\mathfrak{B}}
\newcommand{\frakC}{\mathfrak{C}} 
\newcommand{\frakD}{\mathfrak{D}}
\newcommand{\frakE}{\mathfrak{E}}
\newcommand{\frakF}{\mathfrak{F}} 
\newcommand{\frakG}{\mathfrak{G}}
\newcommand{\frakH}{\mathfrak{H}}
\newcommand{\frakI}{\mathfrak{I}} 
\newcommand{\frakJ}{\mathfrak{J}}
\newcommand{\frakK}{\mathfrak{K}}
\newcommand{\frakL}{\mathfrak{L}} 
\newcommand{\frakM}{\mathfrak{M}}
\newcommand{\frakN}{\mathfrak{N}}
\newcommand{\frakO}{\mathfrak{O}} 
\newcommand{\frakP}{\mathfrak{P}}
\newcommand{\frakQ}{\mathfrak{Q}}
\newcommand{\frakR}{\mathfrak{R}} 
\newcommand{\frakS}{\mathfrak{S}}
\newcommand{\frakT}{\mathfrak{T}}
\newcommand{\frakU}{\mathfrak{U}} 
\newcommand{\frakV}{\mathfrak{V}}
\newcommand{\frakW}{\mathfrak{W}} 
\newcommand{\frakX}{\mathfrak{X}}
\newcommand{\frakY}{\mathfrak{Y}}
\newcommand{\frakZ}{\mathfrak{Z}}

\newcommand{\fraka}{\mathfrak{a}} 
\newcommand{\frakb}{\mathfrak{b}}
\newcommand{\frakc}{\mathfrak{c}} 
\newcommand{\frakd}{\mathfrak{d}}
\newcommand{\frake}{\mathfrak{e}}
\newcommand{\frakf}{\mathfrak{f}} 
\newcommand{\frakg}{\mathfrak{g}}
\newcommand{\frakh}{\mathfrak{h}}
\newcommand{\fraki}{\mathfrak{i}} 
\newcommand{\frakj}{\mathfrak{j}}
\newcommand{\frakk}{\mathfrak{k}}
\newcommand{\frakl}{\mathfrak{l}} 
\newcommand{\frakm}{\mathfrak{m}}
\newcommand{\frakn}{\mathfrak{n}}
\newcommand{\frako}{\mathfrak{o}} 
\newcommand{\frakp}{\mathfrak{p}}
\newcommand{\frakq}{\mathfrak{q}}
\newcommand{\frakr}{\mathfrak{r}} 
\newcommand{\fraks}{\mathfrak{s}}
\newcommand{\frakt}{\mathfrak{t}}
\newcommand{\fraku}{\mathfrak{u}} 
\newcommand{\frakv}{\mathfrak{v}}
\newcommand{\frakw}{\mathfrak{w}} 
\newcommand{\frakx}{\mathfrak{x}}
\newcommand{\fraky}{\mathfrak{y}}
\newcommand{\frakz}{\mathfrak{z}}

\newcommand{\rmA}{\textrm{A}} 
\newcommand{\rmB}{\textrm{B}}
\newcommand{\rmC}{\textrm{C}} 
\newcommand{\rmD}{\textrm{D}}
\newcommand{\rmE}{\textrm{E}}
\newcommand{\rmF}{\textrm{F}} 
\newcommand{\rmG}{\textrm{G}}
\newcommand{\rmH}{\textrm{H}}
\newcommand{\rmI}{\textrm{I}} 
\newcommand{\rmJ}{\textrm{J}}
\newcommand{\rmK}{\textrm{K}}
\newcommand{\rmL}{\textrm{L}} 
\newcommand{\rmM}{\textrm{M}}
\newcommand{\rmN}{\textrm{N}}
\newcommand{\rmO}{\textrm{O}} 
\newcommand{\rmP}{\textrm{P}}
\newcommand{\rmQ}{\textrm{Q}}
\newcommand{\rmR}{\textrm{R}} 
\newcommand{\rmS}{\textrm{S}}
\newcommand{\rmT}{\textrm{T}}
\newcommand{\rmU}{\textrm{U}} 
\newcommand{\rmV}{\textrm{V}}
\newcommand{\rmW}{\textrm{W}} 
\newcommand{\rmX}{\textrm{X}}
\newcommand{\rmY}{\textrm{Y}}
\newcommand{\rmZ}{\textrm{Z}}

\newcommand{\Mu}{\textrm{M}}
\newcommand{\Tau}{\textrm{T}}

\newcommand{\scrA}{\mathscr{A}} 
\newcommand{\scrB}{\mathscr{B}}
\newcommand{\scrC}{\mathscr{C}} 
\newcommand{\scrD}{\mathscr{D}}
\newcommand{\scrE}{\mathscr{E}}
\newcommand{\scrF}{\mathscr{F}} 
\newcommand{\scrG}{\mathscr{G}}
\newcommand{\scrH}{\mathscr{H}}
\newcommand{\scrI}{\mathscr{I}} 
\newcommand{\scrJ}{\mathscr{J}}
\newcommand{\scrK}{\mathscr{K}}
\newcommand{\scrL}{\mathscr{L}} 
\newcommand{\scrM}{\mathscr{M}}
\newcommand{\scrN}{\mathscr{N}}
\newcommand{\scrO}{\mathscr{O}} 
\newcommand{\scrP}{\mathscr{P}}
\newcommand{\scrQ}{\mathscr{Q}}
\newcommand{\scrR}{\mathscr{R}} 
\newcommand{\scrS}{\mathscr{S}}
\newcommand{\scrT}{\mathscr{T}}
\newcommand{\scrU}{\mathscr{U}} 
\newcommand{\scrV}{\mathscr{V}}
\newcommand{\scrW}{\mathscr{W}} 
\newcommand{\scrX}{\mathscr{X}}
\newcommand{\scrY}{\mathscr{Y}}
\newcommand{\scrZ}{\mathscr{Z}}

%%%%%%%%Spacing%%%%%%%%%%%%%%%%%%%
\newcommand{\tab}{\hspace{3ex}}


\newcommand{\bispt}{\Spt_{S^1,\bbG_m}}
\newcommand{\symspt}{\Spt^{\Sigma}}
\newcommand{\Fpbar}{\overline{\bbF}_p}
\newcommand{\cosk}{\mathrm{cosk}}

\begin{document}

\section{Voevodsky's connectivity theorem for $\bbP^1$-spectra}

Our goal is to prove theorem 4.14 of \cite{Voev98}, which we restate
in terms of $\bbP^1$-spectra. 

\begin{theorem}
  \label{Voev}
  Let $(X,x)$ be a pointed smooth scheme over $\Spec(k)$ where $k$ is
  an infinite field. Let $\calY$ be a pointed space. Then for any
  $n > \dim(X)$, and any integer $m$
  \begin{equation*}
    \SH(k)(\Sigma^{\infty}X, S^n\wedge \bbG_m^m \wedge \Sigma^{\infty} \calY) = 0.
  \end{equation*}
\end{theorem}

We will prove this theorem by following \cite{Mor03} and \cite{Mor05}
by Fabien Morel.

\begin{remark}
  To prove this theorem, Morel carefully analyzes how to pass from
  spaces in the projective model structure to the $\bbA^1$ stable
  homotopy category of $\bbP^1$ spectra. From the projective model
  structure on spaces, we construct a model of the left Bousfield
  localization of spaces at the class of maps
  $\{ U_+ \wedge \bbA^1 \to U \st U \im \Sm/k\}$. To get to $\bbP^1$
  spectra, we first invert $S^1 \wedge -$ to get a category of
  $S^1$-spectra, and then we invert $\bbG_m \wedge -$ to get a
  category of $(\bbG_m, S^1)$ bispectra.
  \begin{equation*}
    \calH_{s,\bullet}(k) \to \calH_{\bullet}(k) \to \SH^{S^1}(k) \to \SH(k)
  \end{equation*}

  The line of attack is then to show that for a pointed space $\calY$,
  the $S^1$ suspsension spectrum $\Sigma^{\infty}_s \calY$ in the
  stable model category is $-1$ connected. Then we establish the first
  connectivity theorem which ensures the $\bbA^1$ localization of
  $\Sigma^{\infty}_s\calY$ is $-1$ connected. Finally, we show that
  inverting $\bbG_m$ does not affect the connectivity, i.e.,
  $\Sigma^{\infty}_{t}\Sigma^{\infty}_s \calY$ is again $-1$
  connected.

  The machinery that we set up to prove this theorem will also allow
  us to establish a $t$-structure on $\SH(k)$, and identify its heart.
\end{remark}

\begin{remark}
  A construction of Ayoub \cite{Ayoub} shows that theorem \ref{Voev}
  statement is false over general Noetherian base schemes $S$. The
  argument below works for infinite fields, however. 
\end{remark}

\section{Assumptions from previous lectures}

We briefly recall some of the basic constructions which appear in
\cite{Mor03} and \cite{Mor05}. 

\subsection{Facts about Nisnevich topology}

The proof of Voevodsky's connectivity theorem will follow from the
following property of Nisnevich sheaf cohomology by a sequence of
reductions.

\begin{proposition}\cite[2.4.1]{Mor04}
  Let $M$ be a sheaf of abelian groups on $\Sm/k$, and let
  $X \in \Sm/k$ with Krull dimension $d$. Then whenever $n > d$,
  $H^n_{Nis}(X;M) = 0$.
\end{proposition}

% \begin{proposition}\cite[2.4.1]{Mor04}
%   For any $X \in \Sm/k$, and for any $x \in X(k)$, there is an
%   isomorphism of pointed sheaves of sets in the Nisnevich topology
%   \begin{equation*}
%     X/(X-\{x\}) \cong \bbA^n / (\bbA^n - \{0\}).
%   \end{equation*}
% \end{proposition}

\subsection{Unstable model category $\Delta^{op} \Shv(\Sm/k,Nis)$}

\begin{definition}
  Let $k$ be a field, and let $\Sm/k$ denote the category of smooth
  schemes of finite type over $k$. The category of Morel-Voevodsky
  spaces over $k$ is the category of simplicial Nisnevich sheaves on
  $\Sm/k$. We write $\Spc(k) = \Delta^{op}\Shv(\Sm/k, Nis)$ for this
  category.
\end{definition}

The category $\Spc(k)$ may be equipped with several different model
category structures. We will work with the injective local model
category structure on $\Spc(k)$, which we now define.

\begin{definition}
  A map $\calX \to \calY$ is an injective weak equivalence if and only
  if for any $U \in \Sm/k$, the map $\calX(U) \to \calY(U)$ is a weak
  equivalence of simplicial sets. 

  A map $\calX \to \calY$ is an injective cofibration if and only if
  for any $U \in \Sm/k$, the map $\calX(U) \to \calY(U)$ is a
  cofibration of simplicial sets, i.e., a monomorphism.

  A map $\calX \to \calY$ is an injective fibration if and only if it
  satisfies the left lifting property with respect to any trivial
  injective cofibration. That is, for any commutative square below
  with $\calA \to \calB$ a trivial cofibration, a lift
  $\calB \to \calX$ exists.
  \begin{equation*}
    \xymatrix{
      \calA \ar@{{ >}{-}{>}}[d]_{\sim} \ar[r]& \calX \ar[d] \\
      \calB \ar[r] \ar@{{}{..}{>}}[ur]& \calY
    }
  \end{equation*}

  Denote the homotopy category associated to the injective model
  category structure on $\Spc(k)$ by $\calH_{s}(k)$. The ``s'' stands
  for simplicial. 
\end{definition}

\begin{definition}
  The category of pointed space $\Spc_{\bullet}(k)$ inherits a model
  category structure from $\Spc(k)$. The functor
  $-_{+} : \Spc(k) \to \Spc_{\bullet}(k)$ defined by adding a disjoint
  basepoint to a given space is a left Quillen functor. The right
  adjoint is the forgetful functor.
\end{definition}

\begin{proposition}
  Every object of $\Spc(k)$ and $\Spc_{\bullet}(k)$ is cofibrant in
  the injective model category structure. 
\end{proposition}

\begin{definition}
  For $X \in \Sm/k$, let $rX$ denote the sheaf associated to the
  presheaf $U \mapsto \Sm/k(U,X)$. This defines a functor
  $r : \Sm/k \to \Spc(k)$. 

  For $K$ a simplicial set, the constant space $cK \in \Spc(k)$ is the
  sheaf associated to the constant presheaf with value $K$. The
  functor $c : \sSet \to \Spc(k)$ is a left Quillen functor with right
  adjoint given by taking sections at $\Spec k$.
\end{definition}

\begin{proposition}
  $\Spc(k)$ is a simplicial model category. 
\end{proposition}

See \cite[Chapter 2]{Pelaez} for a detailed treatment of the products
and internal hom constructions in $\Spc(k)$. We recount those
definitions which are essential to our argument. 

\begin{definition}
  For spaces $\calX$ and $\calY$, the product $\calX \times \calY$ in
  $\Spc(k)$ is given by $U \mapsto \calX(U) \times \calY(U)$.  For
  spaces $\calX$ and $\calY$, the internal hom
  $\underline{\Hom}(\calX, \calY)$ in $\Spc(k)$ is given by the
  formula
  \begin{equation*}
    (U,m) \in \Sm/k\times \Delta \mapsto \Hom_{\Delta^{op}\Shv}(X \times rU \times c\Delta^n, Y).
  \end{equation*}
\end{definition}

\begin{proposition}
  The product and internal hom defined above give $\Spc(k)$ the
  structure of a closed monoidal model category. See \cite[Chapter
  4]{H-Mod} or \cite[\S1.7]{Pelaez} for the definition.
\end{proposition}

\begin{proof}
  The adjunction between $\calX \times - $ and
  $\underline{Hom}(\calX, -)$ is given by the following map. 
  \begin{equation*}
    \eta : \Hom(\calY, \underline{Hom}(\calX, \calZ)) \xrightarrow{\cong} \Hom(\calX \times \calY, \calZ) 
  \end{equation*}
  For $g \in \Hom(\calY, \underline{Hom}(\calX, \calZ))$, we define
  $\eta(g)$ by
  \begin{equation*}
    \xymatrix@R=5pt{
    \calX_n(U)\times \calY_n(U) \ar[r]^-{\eta(g)(U,n)} & \calZ_n(U) \\ 
  (a,b) \ar@{{ |}{-}{>}}[r] & g(U,n)(b)(U,n)(a,\id_U,\id_{\Delta^n}).
                 }
  \end{equation*}

  For $f : \calX \times \calY \to \calZ$, the map
  $(\eta^{-1}f)(U,n) : \calY_n(U) \to \underline{Hom}(\calX ,
  \calZ)_n(U)$ is given by sending $y \in \calY_n(U)$ to the map
  \begin{equation*}
    \xymatrix@C=2cm@R=5pt{
      \calX_m(V) \times \Sm/k(V,U) \times \Delta^{n}_m \ar[r]^-{(\eta^{-1}f)(U,n)(y)(V,m)} & \calZ_m(V) \\
      (x,\phi,\alpha) \ar@{{ |}{-}{>}}[r] &f(V,m)(x,\calY(\phi)(y\circ \alpha)) 
    }
  \end{equation*}
  where we identify $y$ with a map $y : \Delta^n \to \calY(V)$, and
  $\alpha : \Delta^m \to \Delta^n$.
\end{proof}

\begin{definition}
  Let $\calX$ and $\calY$ be spaces. For a point $x \in \calX$, there
  is an evaluation map
  $ev_x : \underline{Hom}(\calX, \calY) \to \calY$, where at
  $(U,n) \in (\Sm/k \times \Delta)^{op}$ we send
  $g : \calX \times rU \times c\Delta^n \to \calY$ to
  $g(U,n)(x, \id, \id) \in \calY_n(U)$.

  For pointed spaces $(\calX, x)$ and $\calY, y)$, the pointed
  internal hom $\underline{Hom}_{\bullet}(\calX, \calY)$ is the fiber
  of $ev_x$ over $y$, i.e., $ev_{x}^{-1}(y)$.
\end{definition}

\begin{definition}
  Let $(\calX, x)$ and $(\calY, y)$ be pointed spaces. The wedge of
  $\calX$ and $\calY$, denoted by $\calX \vee \calY$, is the pushout
  of the following diagram.
  \begin{equation*}
    \xymatrix{
      pt \ar[r]^x \ar[d]^y & \calX \ar[d] \\
      \calY \ar[r] & \calX \vee \calY
      }
  \end{equation*}

  The smash product $\calX \wedge \calY$ is the
  space given by the pushout of the following diagram, with basepoint
  $\calX \vee \calY$.
  \begin{equation*}
    \xymatrix{
      \calX \vee \calY \ar[r] \ar[d] & \calX \times \calY \ar[d]\\
      pt \ar[r] & \calX \wedge \calY
      }
  \end{equation*}

\end{definition}


\begin{proposition}
  The category of pointed spaces $\Spc_{\bullet}(k)$ is also a closed
  monoidal category with product $\wedge$ and internal hom
  $\underline{\Hom}_{\bullet}$.
\end{proposition}


\subsection{$\bbA^1$ localization}
%     \item See \cite[Prop 2.3.3.]{Pelaez} for details on the various properties
% of fibrant objects in the unstable motivic category.

\begin{definition} A space $\calX$ is called $\bbA^1$ local if for any
  smooth scheme $U$, the canonical map
  \begin{equation*}
    \Hom(rU, \calX) \to \Hom(rU\times \bbA^1, \calX)
  \end{equation*}
  is a bijection. 
\end{definition}

\begin{definition}
  A map $f: \calX \to \calY$ is an $\bbA^1$ weak equivalence if 
  \begin{equation*}
    \Hom(\calY, \calZ) \to \Hom(\calX, \calZ)
  \end{equation*}
  is a bijection for every $\bbA^1$ local space $\calZ$.
\end{definition}

The unstable motivic homotopy category is obtained by left Bousfield
localization of the injective model category structure on spaces with
respect to the class of maps
$W = W_{\bbA^1} = \{ U \times \bbA^1 \to U \st U \in \Sm/k\}$. We
deonte the category of spaces with the model structure obtained by
left Bousfield localization by $\Spc^{\bbA^1}(k)$ and its homotopy
category by $\calH(k)$. See %\cite[Definition 3.3.1]{Hhorn}
\cite[Chapter 3]{Hhorn} for the general theory of Bousfield
localization. One thing we obtain is a localization functor
$L_{\bbA^1} : \calH_s(k) \to L_{W}\calH_s(k)$ which is a left Quillen
functor.  In particular, $L_{\bbA^1}$ sends sends $\bbA^1$ weak
equivalences to isomorphisms.

The model category $\Spc^{\bbA^1}(k)$ is constructed as follows. The
underlying category of $\Spc^{\bbA^1}(k)$ is $\Spc(k)$, but the weak
equivalences are the $\bbA^1$-local weak equivalences. The
cofibrations are the cofibrations in the injective model structure on
$\Spc(k)$. The fibrations are what they need to be, i.e., those maps
which satisfy the left lifting property with respect to trivial
cofibrations, i.e., $\bbA^1$-local weak equivalences which are also
cofibrations.

%Morel writes $L_{\bbA^1}$ for the (derived) functor which sends a
%space (without base point) $\calX$ to an $\bbA^1$-localizion.

In \cite[Proposition 3.2.3]{Mor05}, Morel gives an explicit
construction which takes a pointed space $\calY$ and produces a space
$L^{\infty}\calY$ which is $\bbA^1$ local and a map
$\calY \to L^{\infty}\calY$ which is an $\bbA^1$ weak equivalence. 

% Let $\calY_f$ be the functorial fibrant replacement of $\calY$ with
% respect to the injective model structure. Consider $\bbA^1$ to be
% pointed at $0$. Morel defines $L^{(1)}(\calY)$ to be the cone of the
% map $ev_1 : \underline{Hom}_{\bullet}(\bbA^1, \calY_f) \to \calY_f$.
% So there is a map $\calY \to L^{(1)}(\calY)$ obtained from the trivial
% cofibration $\calY \to \calY_f$ and the defining map
% $\calY_f \to L^{(1)}(\calY)$.

% So $L^{(1)}(-)$ is a functor with a natural transformation
% $\eta : \id \to L^{(1)}(-)$. Define by induction
% $L^{(n)}(\calY) = L^{(1)}(L^{(n-1)}(\calY))$. There is thus a directed
% system $L^{(n-1)}(\calY) \to L^{(n)}(\calY)$. Denote the direct
% limit/hocolim of this directed system by $L^{\infty}(\calY)$.

% \begin{proposition}
%   The natural morphism $\calY \to L^{\infty}(\calY)$ is an $\bbA^1$
%   weak equivalence, and $L^{\infty}(\calY)$ is $\bbA^1$ local.
% \end{proposition}

\begin{definition}
  Let $\calX$ be a space. Define $\pi_0(\calX)$ to be the
  sheaf on $\Sm/k$ associated to $U \to \pi_0(\calX(U))$. A space
  $\calX$ is called $0$-connected if and only if
  $\pi_0(\calX)$ is the trivial sheaf. 

  Let $(\calX,x)$ be a pointed space. Define
  $\pi_n(\calX)$ to be the sheafification of the presheaf
  on $\Sm/k$ given by
  \begin{equation*}
    U \to \pi_n(\calX(U)).
  \end{equation*}

  A pointed space $\calX$ is called $n$-connected if it is
  $0$-connected and for all $i\leq n$, the sheaves
  $\pi_i(\calX)$ are trivial.
\end{definition}

\begin{proposition}
  Let $\calX$ be a $0$-connected simplicial sheaf. Then
  $L^{\infty}\calX$ is also $0$-connected.
\end{proposition}

For a sheaf of abelian groups $M$ on $\Sm/k$ and a natural number $n$,
a Dold-Kan construction gives a simplifical presheaf $K(M,n)$. It is
called the Eilenberg-MacLane spectrum of type $(M,n)$ and has homotopy
sheaves as expected:
\begin{equation*}
  \pi_m(K(M,n)) = \begin{cases}0 & \text{if }m\neq n \\ M & \text{if } m = n \end{cases}.
\end{equation*}

Important equation for this construction. For $X\in\Sm/k$, $M$ a sheaf
of Abelian groups,
\begin{equation*}
  \calH_s(k)(rX, K(M,n)) \cong H^n_{Nis}(X; M).
\end{equation*}
It therefore follows that
\begin{equation*}
  \calH_{\bullet}(k)(rX_+, K(M,n)) \cong H^n_{Nis}(X; M).
\end{equation*}

Use square brackets to denote the maps in the unstable pointed
(motivic) homotopy category, i.e.,
$[\calX, \calY] = \calH_{\bullet}(k)(\calX, \calY)$.

Use $\pi_n^{\bbA^1}(\calX)$ for the sheaf of homotopy groups in the
motivic category, i.e.,
$\pi^{\bbA^1}_n(\calX) = \pi_n(L^{\infty}\calX)$.  This is also
obtained by sheafifying the presheaf given by
\begin{equation*}
  U \in \Sm/k \mapsto [S^n\wedge U_+, \calX].
\end{equation*}


\subsection{Homotopy purity, connectedness calculations}

\begin{proposition}
  Let $k$ be an infinite field, and consider $\calX$ be a pointed
  space. If for any finitely generated field $F$ over $k$,
  $\pi_0(\calX)(F) = 0$, then the sheaf $\pi_0(\calX)$ is trivial.
\end{proposition}

\begin{proof}
  Check at points. Use Gabber presentation lemma. 
\end{proof}

\subsection{$S^1$-spectra}

\begin{definition}
  Let $\Spt_s(k)$ denote the category of $S^1$-spectra of spaces
  $\Delta^{op}\Shv(\Sm/k)$. We first endow this category with the
  projective model structure (or level-wise model structure), i.e., a
  map $f:E\to F$ is a weak equivalence iff for any $n$ the map
  $f_n : E_n \to F_n$ is a w.e.; a map $f : E\to F$ is a fibration iff
  for all $n$ the map $f_n : E_n \to F_n$ is a fibration. The
  cofibrations are characterized by the property that $f : E \to F$ is
  a cofib iff $f_0 : E_0 \to F_0$ is a cofib and for any $n \geq 1$
  \begin{equation*}
    \xymatrix{
      S^1\wedge E_{n-1} \ar[r]^-{\sigma_{n-1}} \ar[d]_{S^1\wedge f_{n-1}} & E_n \ar@(r,u)[rdd]^{f_n} \ar[d] & \\ 
      S^1 \wedge F_{n-1} \ar[r] \ar@(d,l)[rrd]^{\sigma_{n-1}}& P.O. \ar@{{ >}{..}{>}}[rd]& \\
      && F_n
    } 
  \end{equation*}
\end{definition}

This model structure does not actually invert $S^1 \wedge -$. To
accomplish this, we must localize with respect to the stable
equivalences.

\begin{definition}
  A map $f : E \to F$ of $S^1$-spectra is a stable equivalence if for
  any $n\in \bbZ$ the induced map of homotopy sheaves
  $\pi_n(f) : \pi_n(E) \to \pi_n(F)$ is an isomorphism. 

  The stable model category structure on $\Spt_s(k)$ is given by
  declaring the weak equivalences to be the stable weak equivalences,
  and the cofibrations to be the same as those for the projective
  model structure. This is indeed a left Bousfield localization, but
  we will not describe it further as such. 
\end{definition}


\begin{definition}
  Let $\Spt_s^{\bbA^1}(k)$ denote the category of $S^1$ spectra
  endowed with the stable model category structure localized at the
  collection of maps
  $\{\Sigma^{\infty} U_+ \wedge \bbA^1 \to \Sigma^{\infty}U_+ \st U
  \in \Sm/k\}$. 

  Let $\SH^{S^1}(k)$ denote the homotopy category associated to
  $\Spt_s^{\bbA^1}(k)$. We will use $\SH_s^{S^1}(k)$ to denote the
  homotopy category of $\Spt_s(k)$.
\end{definition}

\begin{remark}
  There is a functor $L^{\infty}$ on the category of $S^1$ spectra
  which is similar to the unstable construction.

  So we can use the functor $L^{\infty}$ as an $\bbA^1$ localization
  functor. To be precise, we let $\Spt_s^{\bbA^1loc}(k)$ be the
  subcategory of $\Spt_s(k)$ consisting of the $\bbA^1$ local
  spectra. We may equip $\Spt_s^{\bbA^1loc}(k)$ with a model structure
  with weak equivalences the $\bbA^1$ weak equivalences, the
  cofibrations are the stable cofibrations, and the fibrations are
  what they have to be.

  The functor $L^{\infty} : \Spt_s(k) \to \Spt_s^{\bbA^1loc}(k)$ is
  left adjoint to the inclusion functor, and is a left Quillen
  functor. The homotopy category of $\Spt_s^{\bbA^1loc}(k)$ is
  categorically equivalent to $\SH^{S^1}(k)$. \cite[Corollary
  4.2.3]{Mor05}.

  For spectra $E$ and $F$, we may compute
  $[E,F]^{\bbA^1}:= \SH^{S^1}(k)(E,F)$ by calculating
  $[L^{\infty} E, L^{\infty} F]^{\bbA^1}$. Note that this is
  $\SH^{S^1}_s(k)(E,L^{\infty}F)$ by using the adjunction. If we
  assume $E$ is cofibrant and $L^{\infty}F$ is fibrant, we get the
  formula
  \begin{equation*}
    [E,F]^{\bbA^1} = \Spt_s(k)(E,L^{\infty}F).
  \end{equation*}
\end{remark}

\begin{definition}
  Let $E$ be an $S^1$ spectrum of spaces. Let $\pi_n$ denote the sheaf
  obtained by taking the colimit of the directed system
  $\pi_{n+r}(E_r)$ in $\Ab(\Sm/k,Nis)$. That is,
  \begin{equation*}
    \pi_n(E) = \colim_{r} \pi_{n+r}(E_r). 
  \end{equation*}

  In particular, for a $U \in \Sm/k$, we have 
  \begin{equation*}
    \pi_n(E)(U) = \colim_{r} \pi_{n+r}(E_r)(U).
  \end{equation*}
\end{definition}

\begin{definition}
  An $S^1$-spectrum $E$ is said to be $n$-connected if for any
  $m\leq n$, the homotopy sheaves $\pi_m(E)$ are trivial.
\end{definition}

\begin{definition}
  There is a left Quillen functor
  $\Sigma^{\infty}_s : \Spc_{\bullet} \to \Spt_s(k)$ given by
  $(\Sigma^{\infty}\calY)_n =(S^1)^{\wedge n} \wedge \calY$ with the
  evident bonding maps. The right adjoint to this functor is given by
  ``evaluation at $0$'', i.e., $\Omega^{\infty}(E) = E_0$.
\end{definition}

\begin{remark}
  The right derived functor
  $R\Omega^{\infty} : \SH^{S^1}_s(k) \to \calH_{\bullet}(k)$ is given by
  the formula
  \begin{equation*}
    R\Omega^{\infty}(E) = \colim_i \Omega_s^{i}E_i.
  \end{equation*}

  This comes from the fact that fibrant $S^1$ spectra are exactly the
  $\Omega$ spectra, and the description of the fibrant replacement
  functor. 
\end{remark}

\begin{remark}
  We also get a left Quillen functor
  $ \Sigma^{\infty}_s : \Spc_{\bullet}^{\bbA^1}(k) \to
  \Spt_s^{\bbA^1}(k)$ given by the same formula as above.

%  What is the fibrant replacement functor in $\Spt_s^{\bbA^1}(k)$?
\end{remark}

\begin{remark}
  The stable homotopy category is symmetric monoidal, with smash
  product $\wedge$ and internal hom $\underline{\Hom}$. Using
  symmetric spectra, one can give these constructions on the category
  of spectra.

  The stable homotopy category is a triangulated category.  
\end{remark}

\begin{proposition}
  Let $U \in \Sm/k$, $n\in \bbZ$, and $M \in \Ab(\Sm/k)$. Then there
  is a canonical isomoprhism
  \begin{equation*}
    H^n_{Nis}(U; M) \to \SH^{S^1}(\Sigma^{\infty}U_+, HM[n]).
  \end{equation*}
\end{proposition}

\begin{proof}
  This is \cite[Lemma 3.2.3]{Mor05}.
\end{proof}

\subsection{$t$-structures}

\begin{definition}
  Let $\frakC$ be a triangulated category. A $t$-structure on $\frakC$
  is a pair of full subcategories $(\frakC_{\geq 0}, \frakC_{\leq 0})$
  which satisfies 
  \begin{enumerate}
      \item For any $X \in \frakC_{\geq 0}$ and any
    $ Y \in \frakC_{\leq 0}$, $\Hom_{\frakC}(X, Y[-1])=0$.
      \item $\frakC_{\geq 0}[1] \subseteq \frakC_{\geq 0}$ and
    $\frakC_{\leq 0}[-1] \subseteq \frakC_{\leq 0}$

      \item for any $X \in \frakC$ there exists a distinguished triangle 
    \begin{equation*}
      Y \to X \to Z \to Y[1]
    \end{equation*}
    for which $Y \in \frakC_{\geq 0}$, $Z\in \frakC_{\leq 0}[-1]$..
  \end{enumerate}

  The heart of a $t$-structure is the full subcategory given by
  $\frakC_{\geq 0} \cap \frakC_{\leq 0}$. 
\end{definition}

\begin{definition}[$t$-structure on $\SH_s^{S^1}(k)$]
  Define $\SH_s^{S^1}(k)_{\geq 0}$ to be the full subcategory of
  $\SH_s^{S^1}(k)$ consisting of objects $E$ such that $\pi_n(E)=0$
  whenver $n<0$.
  
  Define $\SH_s^{S^1}(k)_{\leq 0}$ to be the full subcategory of
  $\SH_s^{S^1}(k)$ consisting of objects $E$ such that $\pi_n(E)=0$
  whenver $n>0$.
\end{definition}

\begin{theorem}
  The triple
  $(\SH_s^{S^1}(k), \SH_s^{S^1}(k)_{\geq 0}, \SH_s^{S^1}(k)_{\leq 0})$
  is a $t$-structure on $\SH_s^{S^1}(k)$. 
\end{theorem}


\begin{remark}
  For a space $\calX$, there is a Postnikov tower associated to it
  \begin{equation*}
    \cdots P^n(\calX)\to P^{n-1}(\calX) \to \cdots \to P^{0}(\calX) \to P^{-1}(\calX)
  \end{equation*}
  constructed in \cite[p. 57]{MV99}. The main construction needed is the
  Moore-Postnikov tower of a simplicial set \cite[VI.3.4]{GJ91}. For a
  simplicial set $K$ and $n\in\bbN$, define
  $K^{(n)} = \im ( K \to \cosk_nK)$. This is a convenient way to
  define the Moore construction. 

  For a space $\calX$, we then define $P^{n}\calX$ to be the space
  given by sheafification of $U \mapsto \calX(U)^{(n)}$.

  Now for $E$ an $S^1$-spectrum, let $E_{\leq 0}$ be the spectrum with
  $(E_{\leq 0})_n = P^n(E_n)$. The bonding maps come from the
  canonical map
  \begin{equation*}
    S^1 \wedge P^n(E_n) \to P^{n+1}(S^1 \wedge E_n).
  \end{equation*}
  See \cite[Lemma 3.2.1]{Mor05} for more on this construction.
\end{remark}

\subsection{Connectivity results}

\begin{proposition}\cite[Lemma4.2.4]{Mor03}
  The functor
  $L^{\infty} : \Spt^{S^1}_s(k) \to \Spt^{S^1}_{s,\bbA^1}(k)$
  identifies the $\bbA^1$-localized $S^1$ stable homotopy category
  with the homotopy category of $\bbA^1$-local $S^1$ spectra.
\end{proposition}

\begin{theorem}[$S^1$ stable connectivity theorem]
  Let $E \in \SH_s^{S^1}(k)$, and suppose that whenever $n < 0$ the
  sheaf $\pi_n E = 0$. Then for all $n<0$, $\pi_n L_{\bbA^1}E = 0$.
\end{theorem}

\begin{theorem}
  The pair $(\SH_{\geq 0}^{S^1}(k), \SH_{\leq 0}^{S^1}(k))$ is a
  $t$-structure on the category $\SH^{S^1}(k)$. 
\end{theorem}

\begin{proof}
  This is just the restriction of the $t$-structure to the
  $\bbA^1$-local objects.
\end{proof}

\begin{definition}
  Strictly $\bbA^1$ invariant sheaf of Abelian groups.

  If $M$ is strictly $\bbA^1$ invariant sheaf of groups, define the
  Eilenberg-MacLane spectrum $HM$ associated to it. 
\end{definition}

\begin{proposition}
  $HM$ is $\bbA^1$ local iff $M$ is strictly $\bbA^1$ invariant.
\end{proposition}

\begin{proposition}
  The heart of the homotopy $t$ structure is equivalent to the
  category of strictly $\bbA^1$ invariant sheaves.
\end{proposition}

\section{Inverting $\bbG_m\wedge -$; $\bbP^1$ spectra}

\subsection{$\bbG_m$ suspension and loops}

We always consider $\bbG_m$ to be pointed at $1$ unless otherwise
specified.

\begin{definition}
  On the category $\Spt_s(k)$ equipped with the motivic stable model
  category structure, there is a functor
  $\Sigma_t(-) = \bbG_m \wedge -$ given by
  $\Sigma_t(E)_n = \bbG_m \wedge E_n$ with the evident structure
  maps. In order for $\Sigma_t$ to be a left Quillen functor, we must
  replace $\bbG_m$ with a cofibrant object. We do this, and work with
  this object instead of $\bbG_m$.

  Smashing with $\bbG_m$ is also a functor on the unstable category of
  pointed spaces, and we give it the same name $\Sigma_t$.
\end{definition}

\begin{definition}
  The functor $\Sigma_t$ on $\Spc_{\bullet}(k)$ has a right adjoint
  denoted $\Omega_t$. It is given by the formula
  $\Omega_t\calX = \underline{\Hom}_{\bullet}(\bbG_m,\calX)$.

  The functor $\Sigma_t$ on $\Spt_s^{S^1}(k)$ also has a right adjoint
  $\Omega_t$ given by the internal hom functor, i.e.,
  $\Omega_tE = \underline{\Hom}(\Sigma^{\infty}\bbG_{m}, E)$.
\end{definition}

\begin{proposition}
  The functor $\Sigma_t$ is a left Quillen functor on $\Spt_s(k)$ and
  on $\Spt_s^{\bbA^1}(k)$. Furthermore, $\Sigma_t$ is an exact (or
  homological) functor on $\SH^{S^1}(k)$, meaning it sends exact
  triangles to exact triangles.
\end{proposition}

\begin{lemma}
  Let $E \in \Spt_s^{\bbA^1}(k)$ be a $-1$ connected spectrum. Then
  $\Sigma_t E$ is again $-1$ connected.
\end{lemma}

\begin{proof}
  The claim is clear when $E = \Sigma^{\infty}_s \calX$ a pointed
  space, since $\Sigma_t E = \Sigma^{\infty}_s \bbG_m \wedge \calX$ is
  still a suspension spectrum, and so $-1$ connected.

  Now consider a general $-1$ connected spectrum $E$. By \cite[Lemma
  3.3.4]{Mor05}, $E$ is weak equivalent to $ \hocolim E^i$ where
  $E^0 = *$, and for each $n$, there is a family
  $X_{\alpha} \in \Sm/k$ and natural numbers $n_{\alpha}\geq 0$ for
  which
  \begin{equation*}
    \vee_{\alpha}\Sigma^{\infty}_s X_{\alpha,+}[n_{\alpha}-1] \to E^{n-1} \to E^{n}
  \end{equation*}
  is an exact triangle. An induction argument establishes that
  $\Sigma_t E^{n}$ is still $-1$ connected for all $n$; hence
  $\Sigma_t E =\hocolim \Sigma_t E^n$ is also $-1$ connected. Should
  $\Sigma_t E$ fail to be $\bbA^1$-local, we may simply apply
  $L^{\infty}$ to get an $\bbA^1$-local representative of
  $\Sigma_t E$. By the connectivity theorem, $L^{\infty}\Sigma_t E$
  will again be $-1$ connected. 
\end{proof}

\subsection{Contraction in $\Ab(\Sm/k,Nis)$, category of pointed
  sheaves of sets}

\begin{definition}
  Let $G$ be sheaf of pointed sets on $\Sm/k$. The contraction of $G$
  is the sheaf $G_{-1}=G_{con}$ given by the formula 
  \begin{equation*}
    U \in \Sm/k \mapsto \ker( G(X\times \bbG_m) \xrightarrow{ev_1} G(X))
  \end{equation*}
  Where the map $ev_1$ is the map induced by
  $ev_1 : \Spec(k) \to \bbG_m$, i.e., $k[x,x^{-1}] \to k$ given by
  $x\mapsto 1$.

  Note that indeed $G_{-1}$ is a sheaf since it is the kernel of the
  morphism of sheaves $G(-) \to G(-\times \bbG_m)$. The sheaf
  $G(-\times \bbG_m)$ may also be written as
  $\underline{\Hom}(\bbG_m, G)$ when we think of $G$ as a
  space.
\end{definition}

\begin{proposition}
  If $G$ is the trivial sheaf of abelian groups, then so is its
  contraction $G_{-1}$.
\end{proposition}

\begin{proposition}
  Contraction is an exact functor on the category
  $\Ab_{st\bbA^1}(\Sm/k,Nis)$.\footnote{I may be able to prove this
    after I establish the homotopy sheaves of $\Omega_t HM$.}

  For any sheaf $G \in \Ab(\Sm/k,Nis)$ and any $X \in \Sm/k$,
  $G(\bbG_m\times X) = G_{-1}(X) \oplus G(X)$.
\end{proposition}


\subsection{Homotopy sheaves of $\underline{Hom}(\bbG_m, E)$}


\begin{proposition}
  If $G$ is a sheaf of Abelian groups, then
  $G_{-1} = \underline{\Hom}_{\bullet}(\bbG_m, G)$.  Hence
  contraction is right adjoint to $- \wedge \bbG_m$. The claim is also
  true for pointed sheaves of sets.
\end{proposition}

\begin{proof}
  For this category, $\underline{\Hom}_{\bullet}(\bbG_m, G)$ and
  $G_{-1}$ both have sections at $X$ given by
  $ker( ev_1 : G(X \times \bbG_m) \to G(X))$. See description of
  pointed internal hom for this. 
\end{proof}

\begin{remark}
  If $G$ is a sheaf of Abelian groups, we may consider $G$ as a space
  by declaring $G_n = G$ for all $n$ and giving identity maps for the
  structure maps. In particular, $G$ is a pointed space at $0$. 

  We can then realize the contraction as a $\bbG_m$ loop space
  $G_{-1}(X) = \underline{\Hom}_{\bullet}(\bbG_m,G)(X)$.
\end{remark}

\begin{remark}
  Construction of canonical map
  $ \pi_n(\underline{\Hom}(\bbG_m,E)) \to \pi_n(E)_{-1}$ for an $S^1$
  spectrum $E$. 

  First observe that for any $U \in \Sm/k$ and any $n \in \bbZ$ there
  is a map
  \begin{equation*}
    \Spt_s(k)(S^n \wedge \Sigma^{\infty}_s U_+ \wedge \Sigma^{\infty} \bbG_m, E) \times \Spc(k)(U,\bbG_m) \to \pi_0(E)(U) 
  \end{equation*}
  given by sending $(f, \alpha)$ to the composition
  \begin{equation*}
    \xymatrix{\Sigma^{\infty}_s U_+ \ar[r]^-{\id \wedge \Sigma^{\infty}_s\alpha}
      & S^n \wedge \Sigma^{\infty}_s U_+ \wedge \Sigma^{\infty}_s \bbG_m  \ar[r]
      & E.
    }
  \end{equation*}
  Hence there is a map of sheaves 
  \begin{equation*}
    \pi_n(\underline{\Hom}(\bbG_m, E)) \times \bbG_m \to \pi_n(E).
  \end{equation*}

  Does this map descend to the smash? Yes, since if either map is a
  constant map, then so is the composition. 

  We thus get a map of sheaves of pointed sets
  \begin{equation*}
    \pi_n(\underline{\Hom}(\bbG_m, E)) \wedge \bbG_m \to \pi_n(E).
  \end{equation*}

  But by the adjunction
  $-\wedge \bbG_m \dashv \underline{\Hom}_{\bullet}(\bbG_m,-)$ on
  $\Spc_{\bullet}(k)$ we have a morphism 
  \begin{equation*}
    \pi_n(\underline{\Hom}(\bbG_m, E)) \to \underline{\Hom}_{\bullet}(\bbG_m, \pi_n(E)) =\pi_n(E)_{-1}.
  \end{equation*}

  Why is it a map of sheaves of abelian groups? 
\end{remark}

\begin{remark}
  If $E = HM$ is an Eilenberg-MacLane spectrum associated to a
  strictly $\bbA^1$ invariant sheaf of abelian groups $M$, we show
  \begin{equation*}
    \pi_n(\underline{\Hom}(\bbG_m, HM)) \to \underline{\Hom}_{\bullet}(\bbG_m, \pi_n(HM)) =\pi_n(HM)_{-1}.
  \end{equation*}
  is an isomorphism by showing $\underline{\Hom}(\bbG_m, HM)$ is an
  Eilenberg-MacLane spectrum. 
\end{remark}

\begin{proposition}
  For $M \in \Ab_{st\bbA^1}(\Sm/k)$, the spectrum
  $\underline{\Hom}(\bbG_m, HM)$ is weak equivalent to $H(M_{-1})$.
\end{proposition}

\begin{proof}
  Since $\bbP^1 = S^1 \wedge \bbG_m$ in $\calH_{\bullet}(k)$, we have
  $\Sigma^{\infty} \bbP^1 [-1] = \Sigma^{\infty} \bbG_m$. Therefore 
  \begin{align*}
    \SH^{S^1}_s(k)(\Sigma^{\infty}\bbG_m, HM[n]) & = \SH^{S^1}_s(k)(\Sigma^{\infty}\bbP^1 [-1], HM[n]) \\
    & = \SH^{S^1}_s(k)(\Sigma^{\infty}\bbP^1, HM[n+1])\\
    & = \tilde{H}^{n+1}_{Nis}(\bbP^1;M).
  \end{align*}
  As the cohomological dimension of $\bbP^1$ is less than or equal to
  1, we then have $\tilde{H}^{n+1}_{Nis}(\bbP^1;M) = 0$ for all
  $n \neq 0$. By $\tilde{H}^{n}_{Nis}(X;M)$ I mean the kernel of
  $ \SH^{S^1}_s(k)(\Sigma^{\infty}X_+, HM[n]) \to
  \SH^{S^1}_s(k)(\Sigma^{\infty}S^0, HM[n])$
  induced by $S^0 \to X_+$, where this is obtained by choosing a point
  in $X(k)$. But what if $X(k)$ is empty?
  \begin{equation*}
    \tilde{H}^n(X;M) \oplus H^n(\Spec(k);M) \cong H^n(X;M).
  \end{equation*}
  So as $M(\Spec k) \cong M(\bbP^1)$ (follows since $M$ is strictly
  $\bbA^1$ invariant)
  $M(\bbP^1) = pullback( M(\bbA^1)\to M(\bbG_m) \rightarrow M(\bbA^1)
  = M(\Spec k)$),
  we see that the only possible value of $n$ this will not vanish at
  is $1$.
  
  Since this vanishes at fields, a base change argument shows that
  indeed the sheaf $\pi_n \underline{\Hom}(\bbG_m, HM)$ is weakly
  trivial when $n\neq 0$. So then it follows that the sheaf is indeed
  trivial by the argument giving weakly n-connected is equivalent to
  being n-connected.\footnote{Reword this to use the modified lemma
    with gabber presentation.}

  We now calculate at $Spec(k)$
  \begin{align*}
    \pi_0 \underline{\Hom}(\bbG_m, HM)(Spec(k))    
    & = \SH^{S^1}_s(k)(\Sigma^{\infty}\bbG_m, HM) \\
    & = \tilde{H}^0(\bbG_m; M) \\
    & = \ker( \SH^{S^1}_s(k)(\Sigma^{\infty}\bbG_{m,+}, HM) \to \SH^{S^1}_s(k)(\Sigma^{\infty}S^0, HM) )\\
    & = M_{-1}(\Spec k)
  \end{align*}

  We now know that the associated homotopy sheaves
  $\pi_n \underline{\Hom}(\bbG_m, HM)$ and $\pi_nH(M_{-1})$ agree for
  all $n$. So they are weak equivalent by \cite[Lemma 3.2.5]{Mor05}.
\end{proof}


\begin{proposition}
  For any spectrum $E \in \SH^{S^1}(k)$, the homotopy sheaves of
  $\underline{\Hom}(\bbG_m, E)$ are calculated by
  $\pi_n(\underline{\Hom}(\bbG_m,E)) \cong \pi_n(E)_{-1}$
\end{proposition}

\begin{proof}
  Reduce to the case of Eilenberg-MacLane spectra by using the
  Postnikov tower. 
\end{proof}

\subsection{Inverting $\bbG_m \wedge -$; $(\bbG_m, S^1)$ bi-spectra}

The functor $\Sigma_t$ on $\Spt_s(k)$ is a left Quillen functor. We
may therefore apply the general machinery of \cite{H-Spectra} to
create a model category in which $\Sigma_t$ is invertible. The
construction of Hovey may be described as $(\bbG_m, S^1)$ bispectra.

\begin{definition}
  A $(\bbG_m, S^1)$ bi-spectrum of spaces over $k$ consists of a
  bigraded family of spaces $E_{i,j}$, $i,j\geq 0$, equipped with
  structure maps $\sigma_{i,j} : S^1 \wedge E_{i,j} \to E_{i,j+1}$ and
  $\mu_{i,j} : \bbG_m \wedge E_{i,j} \to E_{i+1,j}$ for which the
  following diagram commutes .
  \begin{equation*}
\xymatrix{    S^1 \wedge \bbG_m \wedge E_{i,j} \ar[r]^{S^1\wedge \tau_{i,j}} 
    \ar[d]& S^1 \wedge E_{i+1, j} \ar[dd]^{\sigma_{i+1,j}}\\
    \bbG_m \wedge S^1 \wedge E_{i,j} \ar[d]_{\bbG_m\wedge \sigma_{i,j}}& \\
    \bbG_m \wedge E_{i,j+1}\ar[r]^{\mu_{i,j+1}} & E_{i+1,j+1}
}
  \end{equation*}
\end{definition}

\begin{remark}
  Note that a $(\bbG_m,S^1)$ bispectrum is just a $\bbG_m$-spectrum of
  $S^1$-spectra. So we may therefore equip it with the projective
  stable model structure we get from this perspective. We may
  therefore think of a $(\bbG_m, S^1)$ bi-spectrum $E_{i,j}$ as a
  sequence of $S^1$ spectra $E_{i,*}$. 
\end{remark}

\begin{definition}
  Let $E$ be a $(\bbG_m,S^1)$ bispectrum. Define the bigraded stable
  homotopy presheaf $\tilde{\pi}_{n+ m\alpha}$ by the formula
  \begin{equation*}
    U \in \Sm/k \mapsto \colim_r \calH_{\bullet}(k)(S^{n+r}\wedge \bbG_m^{r+m} \wedge U_+, E_{r,r}).
  \end{equation*}

  Morel's notation is $\tilde{\pi}_n(E)_m = \tilde{\pi}_{n-m\alpha}$.
  We may also write
  $\tilde{\pi}_{n,m}(E) = \tilde{\pi}_{n-m+m\alpha}(E)$.

  We denote the associated Nisnevich sheaf by $\pi_{n+m\alpha}(E)$. 
\end{definition}

\begin{proposition}
  If is $E$ a $(\bbG_m, S^1)$ bispectrum, the presheaf of homotopy
  groups may also be calculated as
  \begin{equation*}
    \tilde{\pi}_{n+m\alpha}E(U) = \colim_{s,r} \calH_{\bullet}(k)(\bbG_m^{s+m} \wedge S^{n+r} \wedge U_+, E_{r,s}).
  \end{equation*}
\cite[p 217]{Nordfjordeid}
\end{proposition}

\begin{definition}
  A morphism $f : E \to F$ of $(\bbG_m, S^1)$ bispectra is an $\bbA^1$
  stable weak equivalence if the following induced map is an
  isomorphism for all $U\in\Sm/k$.
  \begin{equation*}
    f_* : \tilde{\pi}_{n+m\alpha}(E)(U) \to \tilde{\pi}_{n+m\alpha}(F)(U)
  \end{equation*}
\end{definition}

\begin{definition}
  A morphism $f : E \to F$ of $(\bbG_m, S^1)$ bispectra is an $\bbA^1$
  stable cofibration if $f_0 : E_{0,*} \to F_{0,*}$ is a cofibration
  of $S^1$ spectra and the map $P.O. \to F_{n+1}$ is a cofibration in
  the following diagram. 
  \begin{equation*}
    \xymatrix{
      \bbG_m \wedge E_{n,*} \ar[r] \ar[d] & E_{n+1,*} \ar@(r,u)[rdd]^{f_{n+1}} \ar[d] & \\ 
      \bbG_m \wedge F_{n,*} \ar[r] \ar@(d,l)[rrd]& P.O. \ar@{{ >}{..}{>}}[rd]& \\
      && F_{n+1,*}
    } 
  \end{equation*}

\end{definition}

\begin{proposition}
  The category $\Spt_{s,t}^{\bbA^1}(k)$ of $(\bbG_m, S^1)$ bispectra with
  $\bbA^1$ stable weak equivalences and $\bbA^1$ stable cofibrations
  is a model category. Denote the associated homotopy category of
  $\Spt_{s,t}^{\bbA^1}(k)$ by $\SH(k)$.
\end{proposition}

\begin{proposition}
  The fibrant bi-spectra are the $\Omega_t$-spectra. \cite[Theorem
  3.4]{H-Spectra}
\end{proposition}

\begin{proposition}
  There is a left Quillen functor
  $\Sigma^{\infty}_t : \Spt_s^{\bbA^1}(k) \to \Spt_{s,t}^{\bbA^1}(k)$
  given by $(\Sigma^{\infty}_tE)_{i,j} = \bbG_m^i\wedge E_j$ with bonding maps 
  \begin{equation*}
    \xymatrix{
      S^1 \wedge \bbG_m^i \wedge E_j \ar[r] & \bbG_m^i \wedge S^1 \wedge E_j \ar[r] & \bbG_m \wedge E_{j+1}
    }
  \end{equation*}
  and 
  \begin{equation*}
    \bbG^m \wedge \bbG_m^i \wedge E_j \to \bbG_m^{i+1} E_j.
  \end{equation*}

  The right adjoint to $\Sigma^{\infty}_t$ is denoted by
  $\Omega^{\infty}_t$ and is given by
  $\Omega^{\infty}_t(E) = E_{0,*}$.

  The right derived functor $R\Omega^{\infty}_t(E)$ is given by the formula 
  \begin{equation*}
    R\Omega^{\infty}_t(E) = \colim_i \Omega_t^i E_{i,*}.
  \end{equation*}
\end{proposition}

\begin{proof}
  
\end{proof}

\subsection{Connectivity of $(\bbG_m, S^1)$ bispectra}

\begin{definition}
  A $(\bbG_m, S^1)$ bispectrum $E$ is said to be $n$-connected if for
  all $k\leq n$ and all $m \in \bbZ$, the homotopy sheaves
  $\pi_{k + m\alpha}E$ vanish.
\end{definition}

\begin{proposition}
  Let $E \in \Spt_s^{\bbA^1}(k)$. Consider
  $\Sigma^{\infty}_t E \in \Spt_{s,t}^{\bbA^1}(k)$. If $E$ is $-1$ connected,
  then so too is $\Sigma^{\infty}_t E$. 
\end{proposition}

\begin{proof}
  We calculate
  \begin{align*}
    \pi_{n+m\alpha}(\Sigma^{\infty}_t E) 
    & = \pi_n(R\Omega^{\infty}_t\Omega_t^m \Sigma^{\infty}_t E) \\ 
    & = \pi_n (\colim_i \Omega_t^{m+i}\Sigma_t^{i} E) \\
    & = \colim_i \pi_n (\Sigma_t^{i}E)_{-(m+i)}\\
    & = 0.
  \end{align*}
  This follows since $\Sigma_tE$ is $-1$ connected whenever $E$ is
  $-1$ connected, and the effect of $\Omega_t^{m+i}$ on homotopy
  sheaves is contraction.
\end{proof}

\subsection{$t$-structure on $\SH(k)$}

\begin{definition}
  Let $\SH(k)_{\geq 0}$ denote the full subcategory of $\SH(k)$ given
  by bispectra $E$ satisfying $\pi_{n+m\alpha}E = 0 $ whenever
  $n < 0$.

  Let $\SH(k)_{\leq 0}$ denote the full subcategory of $\SH(k)$ given
  by bispectra $E$ satisfying $\pi_{n+m\alpha}E = 0$ whenever $n >0$. 
\end{definition}

\begin{definition}
  For a $(\bbG_m, S^1)$ bispectrum $E$, let $E_{\leq 0}$ denote the
  spectrum with $(E_{\leq 0})_n = (E_n)_{\leq 0}$. The bonding maps
  are given by 
  \begin{equation*}
    \bbG_m \wedge P^j(E_{i,j}) \cong P^j( \bbG_m \wedge E_{i,j}) \to P^j( E_{i+1,j}).
  \end{equation*}
  The equivalence
  $\bbG_m \wedge P^j(\calX) \cong P^j(\bbG_m\wedge \calX)$ follows by
  checking on stalks, and the fact that any stalk of $\bbG_m$ is just
  a disjoint union of points. 0 
\end{definition}

\begin{theorem}
  The triple $(\SH(k), \SH(k)_{\geq 0}, \SH(k)_{\leq 0})$ defines a
  $t$-structure.
\end{theorem}

\begin{proof}
What needs to be done: 
\begin{enumerate}
    \item Let $E \in \SH(k)_{\geq 0}$ and $F \in \SH(k)_{\leq 0}$. We
  must show $\SH(k)(E,F[-1])=0$.  When $E$ is in the image of
  $\Sigma^{\infty}_t$, the result follows by using the adjuction
  $\Sigma^{\infty}_t \dashv R\Omega^{\infty}_t$ and using the
  $t$-structure on $S^1$ spectra. In particular, for $U \in \Sm/k$ we
  have
  $\SH(k)(S^n \wedge \bbG_m^m \wedge \Sigma^{\infty} U_+, F[-1])=0$
  for $n\geq 0$ and $m \in \bbZ$.

  For a general $E \in \SH(k)_{\geq 0}$, we may write
  $E = \hocolim E^i$ where the $E^i$ are built up as in
  \cite[3.3.4]{Mor05}, but we allow smashing with $\bbG_m$.
  Precisely, we take $E^0 = pt$, and each $E^i$ is obtained from
  $E^{i-1}$ as the cone of a map
  \begin{equation*}
    \bigvee_{\alpha}S^{n_{\alpha} }\wedge \bbG_m^{m_{\alpha}} \wedge \Sigma^{\infty} X_{\alpha, +} 
    \to E^{i-1}
  \end{equation*}
  for some family of $X_{\alpha} \in \Sm/k$ and indices
  $n_{\alpha}\geq 0$, $m_{\alpha}\in \bbZ$. 

%  by the fact that the spectra
%  $S^n\wedge \bbG_m^m \wedge \Sigma^{\infty} U_+$ for $n\geq 0$,
%  $m \in \bbZ$, and $U\in \Sm/k$ generate $\SH_{\geq 0}(k)$.

  A standard 5-lemma argument using the long exact sequence obtained
  by applying $\SH(k)(-, F[-1])$ to the triangle
  \begin{equation*}
    \vee S^{n_{\alpha} }\wedge \bbG_m^{m_{\alpha}} \wedge \Sigma^{\infty} X_{\alpha, +} \to E^{i-1} \to E^i
  \end{equation*}
  shows that for all $i\in \bbN$, $\SH(k)(E^i,F[-1])=0$. Furthermore,
  these long exact sequences show that for all $i\geq 1$,
  $\SH(k)(E^{i},F[-2]) \to \SH(k)(E^{i-1}, F[-2])$ is
  surjective. Hence $\InverseLimit^{1} \SH(k)(E^{i}, F[-2]) = 0$, and
  so
  \begin{align*}
    \SH(k)(E, F[-1]) & = \SH(k)(\colim E^i, F[-1]) \\
                     & = \InverseLimit \SH(k)(E^i, F[-1]) \\
                     & = 0.
  \end{align*}

  
    \item Let $E \in \SH(k)_{\geq 0}$ and $F \in \SH(k)_{\leq 0}$.
  Show $E[1] \in \SH(k)_{\geq 0}$ and $F[-1] \in \SH(k)_{\leq 0}$.

  This is clear by invertibility of $[1]$ in $\SH(k)$.

    \item Given $E \in \SH(k)$, construct $E_{\geq 0}$ and $E_{<0}$
  which fit into an exact triangle
  \begin{equation*}
    E_{\geq 0} \to E \to E_{\leq -1} \to E_{\geq 0}[1].
  \end{equation*}

  % We can construct the functor $(-)_{\geq 0}$. For a bi-spectrum $E$,
  % we define $(E_{\geq 0})_n = (E_n)_{\geq 0}$, where we are applying
  % the truncation functor for $S^1$-spectra. The bonding maps are
  % obtained by using the isomorphism
  % $\bbG_m \wedge (E_n)_{\geq 0} \to (\bbG_m \wedge E_n)_{\geq 0}$. It
  The functor $(-)_{\leq 0}$ has already been defined. For $k\in\bbZ$,
  let $(-)_{\leq k}$ is a functor on $\Spt_s(k)$ and we may extend it
  to a functor on $\SH(k)$ in the same way as for the case
  $k=0$. Define $E_{\geq 0}$ to be the homotopy fiber of the canonical
  map $E \to E_{\leq -1}$. The long exact sequence of homotopy groups
  shows that $(-)_{\geq 0}$ has the correct homotopy groups. The
  uniqueness of the triangle follows by properties of triangulated
  categories. 
\end{enumerate}
\end{proof}

\subsection{The heart of the $t$-structure on $\SH(k)$}

\begin{definition}
  A homotopy module over $k$ is a pair $(M_*, \mu_*)$ consisting of a
  $\bbZ$ graded strictly $\bbA^1$ invariant sheaf $M_*$ and an
  isomorphism $\mu_n : M_n \cong (M_{n+1})_{-1}$. 
\end{definition}

\begin{lemma}
  If $E$ is a bi-spectrum, then 
  \begin{equation*}
    R\Omega^{\infty}_t E \to \underline{\Hom}(\bbG_m, R\Omega^{\infty}_t(E\wedge \bbG_m))
  \end{equation*}
  is an isomorphism.
\end{lemma}

\begin{proof}
  
\end{proof}

\begin{lemma}
  Let $E \in \SH(k)$. For a fixed $n\in\bbZ$, the collection
  $\pi_{n}(E)_{m}$ forms a homotopy module.
\end{lemma}

\begin{lemma}
  If $(M_*, \mu_*)$ is a homotopy module over $k$, then there is a
  $(\bbG_m, S^1)$ bispectrum $HM_*$ with $(HM_*)_{n,n} = K(M_n,n)$
  with evident structure maps. 
\end{lemma}

\begin{proof}
  
\end{proof}

\begin{theorem}
  The heart of the $t$-structure
  $(\SH(k), \SH(k)_{\geq 0}, \SH(k)_{\leq 0})$ is denoted
  $\pi^{\bbA^1}_*(k)$ and is equivalent to the category of homotopy
  modules. The equivalence is given explicitly by the functors
  $\pi_0(-)_*$ and $H(-)$. 
\end{theorem}

\begin{proof}
  
\end{proof}


\begin{thebibliography}{OOOO}
    \bibitem[A-1974]{Adams} Adams, J.F., {\it Stable Homotopy and
    Generalized Homology.} Chicago Lectures in Mathematics, (1974).

    % \bibitem[A-1965]{Adams-Periodicity} Adams, J.F., {\it A
    % periodicity theorem in homological
    % algebra}. Proc. Camb. Phil. Soc. (1966).

    \bibitem[B]{Blander} Blander, Benjamin, {\it Local Projective
    Model Structures on Simplicial Presheaves}. K-Theory, {\bf 24}
  (2001) 283--301.

    % \bibitem[DI]{DI} Dugger, D.; Isaksen, D., {\it The motivic Adams
    % spectral sequence}.

    % \bibitem[DI-2005]{DI-Cell} Dugger, D.; Isaksen, D., {\it Motivic
    % cell structures.} Alg. Geom. Topol. {\bf 5} (2005), 615--652.

    \bibitem[DHI]{DHI} Dugger, Dan; Hollander, Sharon; Isaksen, Dan,
  {\it Hypercovers and simplicial presheaves}.

    \bibitem[DL{\O}RV]{Nordfjordeid} Dundas, B.; Levine, M.;
  {\O}stv{\ae}r, P.; R\"ondigs, O.; Voevodsky, V., {\it Motivic
    Homotopy Theory}. Springer (2000).

  %   \bibitem[FW]{Fu-Wilson} Fu, K.; Wilson, G., {\it Algorithms
  %   for computing the cohomology of the motivic Steenrod algebra}, in
  % preparation.

  %   \bibitem[G]{Geisser} Geisser, T., {\it Motivic cohomology over
  %   Dedekind rings.} (2004).

  %   \bibitem[G]{Geisser-Handbook} Geisser, T., {\it Motivic
  %   Cohomology, K-Theory, and Topological Cyclic Homology.} Handbook
  % of $K$-Theory, (2005).

  %   \bibitem[GI]{EtaLocal} Guillou, B.; Isaksen, D., {\it The
  %   $\eta$-local motivic sphere}. preprint (2014).

    \bibitem[GJ91]{GJ91} Goerss, Paul; Jardine, John, {\it Simplicial
    Homotopy Theory}. (1991).

    \bibitem[Hir]{Hhorn} Phillip, Hirschhorn, {\it Model Categories
    and Their Localization}. AMS (2003).

    \bibitem[H-Mod]{H-Mod} Hovey, Mark, {\it Model Categories}.
  online preprint (1991).
  
    \bibitem[H-Spt]{H-Spectra} Hovey, Mark, {\it Spectra and symmetric
    spectra in general model categories}. journal? (2001).

  %   \bibitem[HKO]{HKO} Hu, P.; Kriz, I.; Ormsby, K., {\it Remarks on
  %   motivic homotopy theory over algebraically closed fields} (2008).

  %   \bibitem[HK{\O}]{HKOst} Hoyois, Marc; Kelly, Shane; {\O}stv{\ae}r,
  % Paul Arne, {\it The motivic Steenrod algebra in positive
  %   characteristic}. preprint (2013).

  %   \bibitem[HW]{BK} Haesemeyer, Christian; Weibel, Charles. {\it The
  %   Norm Residue Theorem in Motivic Cohomology}, preprint (2014).

    \bibitem[J]{Jardine} Jardine, J.F., {\it Simplicial
    presheaves}. Journal of Pure and Applied Algebra, {\bf 47} (1987)
  35--87.

    % \bibitem[L]{Levine} Levine, M., {\it A comparison of motivic and
    % classical stable homotopy theories}. arXiv:1201.0283v4, (2013).

    % \bibitem[Mc]{McCleary} McCleary, John, {\it A User's Guide to
    % Spectral Sequences}.

    \bibitem[Mor03]{Mor03} Morel, Fabien, {\it An introduction to $\bbA^1$
    homotopy theory}.

    \bibitem[Mor04]{Mor04} Morel, Fabien, {\it On the motivic $\pi_0$
    of the sphere spectrum}. NATO science series.

    \bibitem[Mor05]{Mor05} Morel, Fabien, {\it The stable $\bbA^1$
    connectivity theorems}. preprint (2004).

    \bibitem[MV99]{MV99} Morel, Fabien; Voevodsky, Vladimir, {\it $\bbA^1$-homotopy theory of schemes}. IHES, tome 90 (1999), p. 45--143. 

    % \bibitem[M-2004]{Morel04} Morel, Fabien, {\it Sur les puissances
    % de l'id\'eal fondamental de l'anneau de Witt}.

    % \bibitem[O{\O}-1]{LowDimFields} Ormsby, Kyle; Ostvaer, Paul, {\it
    % Stable Motivic $\pi_1$ of low-dimensional
    % fields}. arxiv:1310.2970v1 (2013).

  %   \bibitem[O{\O}-2]{MotBPInvQ} Ormsby, Kyle; Ostvaer, Paul, {\it
  %   Motivic Brown-Peterson invariants of the rationals}. arxiv
  % 1208.5007v2 (2013).

  %   \bibitem[O]{MotInvPadic} Ormsby, Kyle, {\it Motivic invariants of
  %   p-adic fields}. (2010).

    \bibitem[Pel08]{Pelaez} Pelaez, Pablo, {\it Multiplicative Properties
    of the Slice Filtration}.  December 2008.

  %   \bibitem[R]{Ravenel} Ravenel, D., {\it Complex Cobordism and
  %   Stable Homotopy Groups of Spheres.} 2nd Edition, AMS (2004).

  %   \bibitem[R{\O}]{ROst} R{\"o}ndigs, O.; {\O}stv{\ae}r, P., {\it
  %   Rigidity in motivic homotopy theory.} preprint? (2007).

  %   \bibitem[So]{Soule} Soul{\'e}, C., {\it $K$-Th{\'e}orie des
  %   anneaux d'entiers de corps de nombres et cohomologie {\'e}tale.}
  % Inventiones math. (1979).

  %   \bibitem[Sp]{Spitzweck} Spitzweck, Markus, {\it A commutative
  %   $\bbP^1$-spectrum representing motivic cohomology over Dedekind
  %   domains}. preprint, (2013).

  %   \bibitem[V]{Voev} Voevodsky, Vladimir., {\it The Milnor
  %   Conjecture}. December 1996. 

    \bibitem[Voev98]{Voev98} Voevodsky, Vladimir. {\it $\bbA^1$-Homotopy
    Theory}. Doc. Math. J., (1998) pp. 579--604.

\end{thebibliography}
\end{document}
